\documentclass[]{article}
\usepackage{lmodern}
\usepackage{amssymb,amsmath}
\usepackage{ifxetex,ifluatex}
\usepackage{fixltx2e} % provides \textsubscript
\ifnum 0\ifxetex 1\fi\ifluatex 1\fi=0 % if pdftex
  \usepackage[T1]{fontenc}
  \usepackage[utf8]{inputenc}
\else % if luatex or xelatex
  \ifxetex
    \usepackage{mathspec}
  \else
    \usepackage{fontspec}
  \fi
  \defaultfontfeatures{Ligatures=TeX,Scale=MatchLowercase}
\fi
% use upquote if available, for straight quotes in verbatim environments
\IfFileExists{upquote.sty}{\usepackage{upquote}}{}
% use microtype if available
\IfFileExists{microtype.sty}{%
\usepackage{microtype}
\UseMicrotypeSet[protrusion]{basicmath} % disable protrusion for tt fonts
}{}
\usepackage[margin=1in]{geometry}
\usepackage{hyperref}
\hypersetup{unicode=true,
            pdfborder={0 0 0},
            breaklinks=true}
\urlstyle{same}  % don't use monospace font for urls
\usepackage{graphicx,grffile}
\makeatletter
\def\maxwidth{\ifdim\Gin@nat@width>\linewidth\linewidth\else\Gin@nat@width\fi}
\def\maxheight{\ifdim\Gin@nat@height>\textheight\textheight\else\Gin@nat@height\fi}
\makeatother
% Scale images if necessary, so that they will not overflow the page
% margins by default, and it is still possible to overwrite the defaults
% using explicit options in \includegraphics[width, height, ...]{}
\setkeys{Gin}{width=\maxwidth,height=\maxheight,keepaspectratio}
\IfFileExists{parskip.sty}{%
\usepackage{parskip}
}{% else
\setlength{\parindent}{0pt}
\setlength{\parskip}{6pt plus 2pt minus 1pt}
}
\setlength{\emergencystretch}{3em}  % prevent overfull lines
\providecommand{\tightlist}{%
  \setlength{\itemsep}{0pt}\setlength{\parskip}{0pt}}
\setcounter{secnumdepth}{0}
% Redefines (sub)paragraphs to behave more like sections
\ifx\paragraph\undefined\else
\let\oldparagraph\paragraph
\renewcommand{\paragraph}[1]{\oldparagraph{#1}\mbox{}}
\fi
\ifx\subparagraph\undefined\else
\let\oldsubparagraph\subparagraph
\renewcommand{\subparagraph}[1]{\oldsubparagraph{#1}\mbox{}}
\fi

%%% Use protect on footnotes to avoid problems with footnotes in titles
\let\rmarkdownfootnote\footnote%
\def\footnote{\protect\rmarkdownfootnote}

%%% Change title format to be more compact
\usepackage{titling}

% Create subtitle command for use in maketitle
\newcommand{\subtitle}[1]{
  \posttitle{
    \begin{center}\large#1\end{center}
    }
}

\setlength{\droptitle}{-2em}
  \title{}
  \pretitle{\vspace{\droptitle}}
  \posttitle{}
  \author{}
  \preauthor{}\postauthor{}
  \date{}
  \predate{}\postdate{}


\begin{document}

\section{Literature Review}\label{literature-review}

\subsection{Two perspectives on Corporate Environmental
Performance}\label{two-perspectives-on-corporate-environmental-performance}

The paradigm of profit maximization of @MiltonFriedman1970 have been
widely challenged these last decades. Whereas @MiltonFriedman1970
considers investment in pollution efficient technology as deviation from
the wealth maximization goal, the literature is showing growing
evidences that improving a company's environmental performance can lead
to better economic or financial performance, and not necessarily to an
increase in cost. @Ambec2008 have demonstrated that the expenses
incurred to reduce pollution can be partly or completely offset by gains
made elsewhere. @Porter1995 argued that rather than simply adding to
cost, properly crafted environmental standards can trigger innovation
offsets, allowing companies to improve their resource productivity. He
even redefined the self concept of value creation in advocating that the
solution lies in the principle of shared value which involves creating
economic value in a way that also creates value for society by
addressing its needs and challenges {[}@Porter2011b, @Porter2018{]}.
@Freeman1984 call to a radical rethinking of our model of the firm.
According to him, companies have to consider their stakeholder, namely
\emph{``any group or individual who can affect or is affected by the
achievement of an organisation's objectives''} (p.25) or else face
negative confrontation from non-shareholder groups, which can lead to
diminished shareholder value, through boycotts, lawsuits and protests
etc. In other words, @Freeman1984 summarizes the idea that companies
should consider corporate environmental performance as unavoidable cost
of doing business.

\subsection{Does it pay to be green?}\label{does-it-pay-to-be-green}

While more and more companies are embracing this new paradigm and
develop profitable business strategies that deliver tangible social
benefits, others keep the old fashion way of @MiltonFriedman1970. This
dichotomy have interested scholars and since they have sought to
empirically answer the question, \emph{``Does it pay to be green?''}. In
a competitive business world, answering this question is crucial to
provide a genuine economic justification to the new paradigm
{[}@Ludecadedebatenexus2014{]}. Although results are mixed, the large
quantity of studies on the nexus between Corporate Environmental
Performance (i.e.~CEP) and Corporate Financial Performance (i.e.~CFP) in
the last two decades allowed the appearance of recent meta-analyses
\footnote{Initially, the literature focused on the link between
  Corporate Social Performance (i.e.~CSP) and Corporate Financial
  Performance (i.e.~CFP). @Orlitzky2001 were the first to consider CEP
  as apart from CSP. Given that @Busch2018 could not detect
  statistically significant differences between the effects of
  environmental CEP and social-related CSP on CFP and concludes that
  good CSP pays off, whether social or environmental related, this study
  considers CSP equals to CEP.} {[}@Orlitzky2001, @Orlitzky2003,
@Wu2006, @Albertini2013, @Dixon-Fowler2013,
@EndrikatMakingsenseconflicting2014, @Ludecadedebatenexus2014,
@WangMetaAnalyticReviewCorporate2016, @Busch2018{]} and all suggest that
indeed it pays to be green. More precisely, a positive and bidirectional
relationship does exist between CEP and CFP meaning that successful
firms may have the resources necessary to improve their environmental
performance, which in turn increases financial benefits that again can
be invested back into further improvements of CEP
{[}@EndrikatMakingsenseconflicting2014{]}.

\subsection{CEP and CFP as a broad
meta-construct}\label{cep-and-cfp-as-a-broad-meta-construct}

CFP is a broad meta-constructs and the current literature have shown
that each construct play a moderator role in the relationship between
CEP and CFP {[}@Orlitzky2003, @Ludecadedebatenexus2014,
@Busch2018{]}.Scholars have mainly adopted three broad subdivisions of
CFP: market-based (investor returns), accounting-based (accounting
returns), and perceptual (survey) measures. Market-based measures
(e.g.~price-earning ratio, Tobin's Q, or share price appreciation)
consider that returns should be measured from the perspective of the
shareholders {[}@Cochran1984a{]}. Accounting-based measures require
profitability and asset utilization indicators such as Return on Asset
(i.e.~ROA) or Return on Equity (i.e.~ROE) {[}@Cochran1984a, @Wu2006{]}.
Finally perceptual measures of CFP is a more subjective approach based
on the perception of survey respondents {[}@Ludecadedebatenexus2014{]}.

CEP is also a broad meta-constructs and no common definition exist in
the literature {[}@Albertini2013,
@EndrikatMakingsenseconflicting2014{]}. Scholars have used a wide
variety of indicators as proxies for approaching the green performance
of companies. @Albertini2013 use a three-group classification to
summarize CEP measures : (i) Environmental Management Measures
(i.e.~EMV) which mostly refer to environmental strategy, integration of
environmental issues into strategic planning processes, environmental
practices, process-driven initiatives, product-driven management
systems, ISO 14001 certification, environmental management system
adoption, and participation in voluntary programs {[}@Molina-Azorin2009,
@Schultze2012{]}. (ii) Environmental Performance Variables (i.e.~EPV)
which are mostly measures quantified in physical units (carbon dioxide
emissions, physical waste, water consumption, toxic release) that can be
positive (emission reduction) or negative (emission generated)
{[}@Albertini2013{]}. (iii) Environmental Disclosure Variables
(i.e.~EDV) such as information releases regarding toxic emission
{[}@Hamilton1995{]}, environmental awards
{[}@Chencrosscountrycomparisongreen2018{]}, environmental accidents and
crises {[}@Blacconiere1994{]}, and environmental investment
announcements {[}@Gilley2000{]}. @EndrikatMakingsenseconflicting2014
split up CEP into two sub-dimensions, namely (i) process-based CEP which
can be linked to the EMV approach of @Albertini2013 and (ii)
outcome-based CEP which can be linked to the EPV dimension. According to
@Xie2007, process-based CEP can be considered as a preliminary step of
outcome-based CEP. Scholars demonstrated that the first approach have a
positive impact on the second one which in turn has a positive impact on
financial performance {[}@Li2017,
@Chencrosscountrycomparisongreen2018{]}.

Although recent recent meta-analyses {[}@Orlitzky2001, @Orlitzky2003,
@Wu2006, @Albertini2013, @Dixon-Fowler2013,
@EndrikatMakingsenseconflicting2014, @Ludecadedebatenexus2014,
@WangMetaAnalyticReviewCorporate2016, @Busch2018{]} have demonstrated
the positive link between CEP and CFP, some scholars advanced that the
multidimensionality of both CEP and CFP constructs are one reason why
the conclusion of the relationship between CEP and CFP have been so
mixed {[}@Albertini2013, @EndrikatMakingsenseconflicting2014,
@MiroshnychenkoGreenpracticesfinancial2017{]}. For instance, @Busch2011a
found that process-based CEP (in terms of carbon management) negatively
affects CFP, while outcome-based CEP (in terms of carbon emissions) has
a positive influence on CFP. @Cavaco2014 and
@Muhammadrelationshipenvironmentalperformance2015 have used both
accounting-based indicators (i.e.~ROA) and market-based indicators
(i.e.~Tobin's Q) as a proxy for CFP and got a positive relation between
ROA and CEP while no relation between Tobin's Q and CEP. A general
consensus have shown that accounting-based CFP are characterized by a
stronger relation to CEP than market-based and perceptual indicators
{[}@Orlitzky2003, @Wu2006, @Albertini2013, @Ludecadedebatenexus2014,
@Busch2018{]}.

Considering the varying findings with regards to process-based CEP and
outcome-based CEP and with a motivation to answer the call of
@EndrikatMakingsenseconflicting2014, I hypothesize the following :

\textbf{Hypothesis 1.} Process-based CEP have a positive impact on
Outcome-based CEP

\textbf{Hypothesis 2.} Outcome-based CEP have a positive impact on CFP

\textbf{Hypothesis 3.} Process-based CEP have a positive impact on CFP

\subsection{When does it pay to be
green?}\label{when-does-it-pay-to-be-green}

@Griffin1997 was the first to call for research that looks at the
CEP-CFP relation over time. While scholars had been mainly answering the
question \emph{``Does it pay to be green?''} some have recently tried to
move forward and gained interest in answering the call of @Griffin1997
with the following question : \emph{``When does it pay to be green?''}
{[}@ManriqueAnalyzingEffectCorporate2017{]}.

@Zhang2017 have shown that CEP has a negative relationship with
short-term financial performance and a positive relationship with
long-term CFP. @Delmas2015 observed that the more a firm decreases
carbon emissions the more positive the investors perceptions of future
market performance and the lower its short term financial performance.
@SongCanenvironmentalmanagement2017 have shown that corporate
environmental management has a significant positive correlation with
future financial performance, however it has no significant correlation
with current financial performance.
@ManriqueAnalyzingEffectCorporate2017 demonstrated that in times of
economic crisis firms which improve their corporate environmental
performance improve their corporate financial performance, this effect
being weaker for firms in developed countries, where only the short-term
corporate financial performance improves, than for firms in emerging and
developing countries, where the short- and long-term corporate financial
performance improve. @Chencrosscountrycomparisongreen2018 have shown
that a firms green performance not only impact an organization's
financial performance in that particular year but also impact the year
that follows.

Those empirical results provide evidences that no common consensus have
be found yet to answer the question \emph{``When does it pay to be
green?''}. @Busch2018 demonstrated that at a meta-research level, the
evidence of a time dependency on the CEP-CFP link is not significant and
that the call of @Griffin1997 remains to date unanswered.

To capture the time dimension in the CFP-CEP nexus, scholars consider
accounting-based measures as a proxy for short term CFP and market-based
measures as a proxy for long term CFP
{[}@EndrikatMakingsenseconflicting2014, @Delmas2015, @Zhang2017,
@ManriqueAnalyzingEffectCorporate2017,
@MiroshnychenkoGreenpracticesfinancial2017{]}. According to
@EndrikatMakingsenseconflicting2014 :

\begin{quote}
\emph{``While accounting-based measures may capture immediate impacts,
they may not appropriately account for intangible and long-term effects
which are likely to be involved in the CEP-CFP link. Market-based
measures, on the other hand, integrate estimations of a firm's future
prospects and reflect the notion of external stakeholders (primarily
investors) {[}@Orlitzky2003, @Peloza2009, @Delmas2011{]}. Thus,
market-based measures may better to capture the long-term value of
certain environmental activities.''}
\end{quote}

Taking into account theoretical arguments and empirical findings and in
order to move forward in answering the call of @Griffin1997, I
hypothesize the following :

\textbf{Hypothesis 4.} CEP have a stronger impact on short term CFP than
on long term CFP


\end{document}
