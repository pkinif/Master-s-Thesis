\documentclass[]{article}
\usepackage{lmodern}
\usepackage{amssymb,amsmath}
\usepackage{ifxetex,ifluatex}
\usepackage{fixltx2e} % provides \textsubscript
\ifnum 0\ifxetex 1\fi\ifluatex 1\fi=0 % if pdftex
  \usepackage[T1]{fontenc}
  \usepackage[utf8]{inputenc}
\else % if luatex or xelatex
  \ifxetex
    \usepackage{mathspec}
  \else
    \usepackage{fontspec}
  \fi
  \defaultfontfeatures{Ligatures=TeX,Scale=MatchLowercase}
\fi
% use upquote if available, for straight quotes in verbatim environments
\IfFileExists{upquote.sty}{\usepackage{upquote}}{}
% use microtype if available
\IfFileExists{microtype.sty}{%
\usepackage{microtype}
\UseMicrotypeSet[protrusion]{basicmath} % disable protrusion for tt fonts
}{}
\usepackage[margin=1in]{geometry}
\usepackage{hyperref}
\hypersetup{unicode=true,
            pdfborder={0 0 0},
            breaklinks=true}
\urlstyle{same}  % don't use monospace font for urls
\usepackage{color}
\usepackage{fancyvrb}
\newcommand{\VerbBar}{|}
\newcommand{\VERB}{\Verb[commandchars=\\\{\}]}
\DefineVerbatimEnvironment{Highlighting}{Verbatim}{commandchars=\\\{\}}
% Add ',fontsize=\small' for more characters per line
\usepackage{framed}
\definecolor{shadecolor}{RGB}{248,248,248}
\newenvironment{Shaded}{\begin{snugshade}}{\end{snugshade}}
\newcommand{\KeywordTok}[1]{\textcolor[rgb]{0.13,0.29,0.53}{\textbf{#1}}}
\newcommand{\DataTypeTok}[1]{\textcolor[rgb]{0.13,0.29,0.53}{#1}}
\newcommand{\DecValTok}[1]{\textcolor[rgb]{0.00,0.00,0.81}{#1}}
\newcommand{\BaseNTok}[1]{\textcolor[rgb]{0.00,0.00,0.81}{#1}}
\newcommand{\FloatTok}[1]{\textcolor[rgb]{0.00,0.00,0.81}{#1}}
\newcommand{\ConstantTok}[1]{\textcolor[rgb]{0.00,0.00,0.00}{#1}}
\newcommand{\CharTok}[1]{\textcolor[rgb]{0.31,0.60,0.02}{#1}}
\newcommand{\SpecialCharTok}[1]{\textcolor[rgb]{0.00,0.00,0.00}{#1}}
\newcommand{\StringTok}[1]{\textcolor[rgb]{0.31,0.60,0.02}{#1}}
\newcommand{\VerbatimStringTok}[1]{\textcolor[rgb]{0.31,0.60,0.02}{#1}}
\newcommand{\SpecialStringTok}[1]{\textcolor[rgb]{0.31,0.60,0.02}{#1}}
\newcommand{\ImportTok}[1]{#1}
\newcommand{\CommentTok}[1]{\textcolor[rgb]{0.56,0.35,0.01}{\textit{#1}}}
\newcommand{\DocumentationTok}[1]{\textcolor[rgb]{0.56,0.35,0.01}{\textbf{\textit{#1}}}}
\newcommand{\AnnotationTok}[1]{\textcolor[rgb]{0.56,0.35,0.01}{\textbf{\textit{#1}}}}
\newcommand{\CommentVarTok}[1]{\textcolor[rgb]{0.56,0.35,0.01}{\textbf{\textit{#1}}}}
\newcommand{\OtherTok}[1]{\textcolor[rgb]{0.56,0.35,0.01}{#1}}
\newcommand{\FunctionTok}[1]{\textcolor[rgb]{0.00,0.00,0.00}{#1}}
\newcommand{\VariableTok}[1]{\textcolor[rgb]{0.00,0.00,0.00}{#1}}
\newcommand{\ControlFlowTok}[1]{\textcolor[rgb]{0.13,0.29,0.53}{\textbf{#1}}}
\newcommand{\OperatorTok}[1]{\textcolor[rgb]{0.81,0.36,0.00}{\textbf{#1}}}
\newcommand{\BuiltInTok}[1]{#1}
\newcommand{\ExtensionTok}[1]{#1}
\newcommand{\PreprocessorTok}[1]{\textcolor[rgb]{0.56,0.35,0.01}{\textit{#1}}}
\newcommand{\AttributeTok}[1]{\textcolor[rgb]{0.77,0.63,0.00}{#1}}
\newcommand{\RegionMarkerTok}[1]{#1}
\newcommand{\InformationTok}[1]{\textcolor[rgb]{0.56,0.35,0.01}{\textbf{\textit{#1}}}}
\newcommand{\WarningTok}[1]{\textcolor[rgb]{0.56,0.35,0.01}{\textbf{\textit{#1}}}}
\newcommand{\AlertTok}[1]{\textcolor[rgb]{0.94,0.16,0.16}{#1}}
\newcommand{\ErrorTok}[1]{\textcolor[rgb]{0.64,0.00,0.00}{\textbf{#1}}}
\newcommand{\NormalTok}[1]{#1}
\usepackage{graphicx,grffile}
\makeatletter
\def\maxwidth{\ifdim\Gin@nat@width>\linewidth\linewidth\else\Gin@nat@width\fi}
\def\maxheight{\ifdim\Gin@nat@height>\textheight\textheight\else\Gin@nat@height\fi}
\makeatother
% Scale images if necessary, so that they will not overflow the page
% margins by default, and it is still possible to overwrite the defaults
% using explicit options in \includegraphics[width, height, ...]{}
\setkeys{Gin}{width=\maxwidth,height=\maxheight,keepaspectratio}
\IfFileExists{parskip.sty}{%
\usepackage{parskip}
}{% else
\setlength{\parindent}{0pt}
\setlength{\parskip}{6pt plus 2pt minus 1pt}
}
\setlength{\emergencystretch}{3em}  % prevent overfull lines
\providecommand{\tightlist}{%
  \setlength{\itemsep}{0pt}\setlength{\parskip}{0pt}}
\setcounter{secnumdepth}{0}
% Redefines (sub)paragraphs to behave more like sections
\ifx\paragraph\undefined\else
\let\oldparagraph\paragraph
\renewcommand{\paragraph}[1]{\oldparagraph{#1}\mbox{}}
\fi
\ifx\subparagraph\undefined\else
\let\oldsubparagraph\subparagraph
\renewcommand{\subparagraph}[1]{\oldsubparagraph{#1}\mbox{}}
\fi

%%% Use protect on footnotes to avoid problems with footnotes in titles
\let\rmarkdownfootnote\footnote%
\def\footnote{\protect\rmarkdownfootnote}

%%% Change title format to be more compact
\usepackage{titling}

% Create subtitle command for use in maketitle
\newcommand{\subtitle}[1]{
  \posttitle{
    \begin{center}\large#1\end{center}
    }
}

\setlength{\droptitle}{-2em}
  \title{}
  \pretitle{\vspace{\droptitle}}
  \posttitle{}
  \author{}
  \preauthor{}\postauthor{}
  \date{}
  \predate{}\postdate{}

\usepackage{color}

\begin{document}

\pagecolor{apricot} This appendix presents the R code used to identify
and remove outliers from the database. This R scipt is the one contains
in the makefile :
\emph{Analysis/DataBase/MakeFile\_RemoveOutliers\_Lag1.Rmd}. This step
had been repeated three times : (i) when dependent variables were lagged
one year (see section: {[}The impact of CEP on CFP{]}) and (ii) two
years behind others variables and (iii) when the GreenScore variables
was the only independent variables considered into the econometric model
(see section: {[}Sensitivity Analysis{]}).

\begin{Shaded}
\begin{Highlighting}[]
\CommentTok{# Packages loading}
\ControlFlowTok{if}\NormalTok{ (}\OperatorTok{!}\KeywordTok{require}\NormalTok{(}\StringTok{"dplyr"}\NormalTok{)) }\KeywordTok{install.packages}\NormalTok{(}\StringTok{"dplyr"}\NormalTok{)}
\KeywordTok{library}\NormalTok{(dplyr)}
\ControlFlowTok{if}\NormalTok{ (}\OperatorTok{!}\KeywordTok{require}\NormalTok{(}\StringTok{"grDevices"}\NormalTok{)) }\KeywordTok{install.packages}\NormalTok{(}\StringTok{"grDevices"}\NormalTok{)}
\KeywordTok{library}\NormalTok{(grDevices)}
\ControlFlowTok{if}\NormalTok{ (}\OperatorTok{!}\KeywordTok{require}\NormalTok{(}\StringTok{"data.table"}\NormalTok{)) }\KeywordTok{install.packages}\NormalTok{(}\StringTok{"data.table"}\NormalTok{)}
\KeywordTok{library}\NormalTok{(data.table)}
\ControlFlowTok{if}\NormalTok{ (}\OperatorTok{!}\KeywordTok{require}\NormalTok{(}\StringTok{"formatR"}\NormalTok{)) }\KeywordTok{install.packages}\NormalTok{(}\StringTok{"formatR"}\NormalTok{)}
\KeywordTok{library}\NormalTok{(formatR)}
\ControlFlowTok{if}\NormalTok{ (}\OperatorTok{!}\KeywordTok{require}\NormalTok{(}\StringTok{"highlight"}\NormalTok{)) }\KeywordTok{install.packages}\NormalTok{(}\StringTok{"highlight"}\NormalTok{)}
\KeywordTok{library}\NormalTok{(highlight)}
\end{Highlighting}
\end{Shaded}

\begin{Shaded}
\begin{Highlighting}[]
\CommentTok{# Database Loading}
\NormalTok{path <-}\StringTok{ "Analysis/DataBase/DataSynchronization/Lag1.csv"}
\NormalTok{Lag1 <-}\StringTok{ }\KeywordTok{read.csv}\NormalTok{(}\DataTypeTok{file =}\NormalTok{ path, }\DataTypeTok{header =} \OtherTok{TRUE}\NormalTok{, }\DataTypeTok{stringsAsFactors =} \OtherTok{FALSE}\NormalTok{)}
\end{Highlighting}
\end{Shaded}

\begin{Shaded}
\begin{Highlighting}[]
\CommentTok{# Select only variables that I need for my model}
\NormalTok{ModelLag1 <-}\StringTok{ }\NormalTok{Lag1 }\OperatorTok\StringTok{ }\KeywordTok{select}\NormalTok{(}\KeywordTok{c}\NormalTok{(YearIndex, CompaniesIndex, }
\NormalTok{    Roa, TobinsQ, DebtToEquityRatio, NetMargin, TotalAssets, }
\NormalTok{    GicsClassification, CarbonProductivity, WaterProductivity, }
\NormalTok{    WasteProductivity, SustainabilityPayLink, SustainableThemedCommitment, }
\NormalTok{    AuditScore))}
\CommentTok{# I transform the 'TotalAssets' column into FirmSize}
\CommentTok{# using the log of TotalAssets}
\NormalTok{ModelLag1}\OperatorTok{$}\NormalTok{TotalAssets <-}\StringTok{ }\KeywordTok{log10}\NormalTok{(ModelLag1}\OperatorTok{$}\NormalTok{TotalAssets)}
\CommentTok{# I use the natural log for TobinsQ}
\NormalTok{ModelLag1}\OperatorTok{$}\NormalTok{TobinsQ <-}\StringTok{ }\KeywordTok{log10}\NormalTok{(ModelLag1}\OperatorTok{$}\NormalTok{TobinsQ)}
\CommentTok{# I rename some columns}
\NormalTok{ModelLag1 <-}\StringTok{ }\NormalTok{ModelLag1 }\OperatorTok\StringTok{ }\KeywordTok{setnames}\NormalTok{(}\DataTypeTok{old =} \KeywordTok{c}\NormalTok{(}\StringTok{"DebtToEquityRatio"}\NormalTok{, }
    \StringTok{"TotalAssets"}\NormalTok{, }\StringTok{"GicsClassification"}\NormalTok{, }\StringTok{"NetMargin"}\NormalTok{), }\DataTypeTok{new =} \KeywordTok{c}\NormalTok{(}\StringTok{"Leverage"}\NormalTok{, }
    \StringTok{"FirmSize"}\NormalTok{, }\StringTok{"Industry"}\NormalTok{, }\StringTok{"Growth"}\NormalTok{))}
\CommentTok{# I define my models in lm as cooks.distance do not}
\CommentTok{# support plm object}
\NormalTok{Roa <-}\StringTok{ }\KeywordTok{lm}\NormalTok{(Roa }\OperatorTok{~}\StringTok{ }\NormalTok{SustainabilityPayLink }\OperatorTok{+}\StringTok{ }\NormalTok{SustainableThemedCommitment }\OperatorTok{+}\StringTok{ }
\StringTok{    }\NormalTok{AuditScore }\OperatorTok{+}\StringTok{ }\NormalTok{CarbonProductivity }\OperatorTok{+}\StringTok{ }\NormalTok{WaterProductivity }\OperatorTok{+}\StringTok{ }
\StringTok{    }\NormalTok{WasteProductivity }\OperatorTok{+}\StringTok{ }\NormalTok{FirmSize }\OperatorTok{+}\StringTok{ }\NormalTok{Growth }\OperatorTok{+}\StringTok{ }\NormalTok{Leverage }\OperatorTok{+}\StringTok{ }\NormalTok{Industry, }
    \DataTypeTok{data =}\NormalTok{ ModelLag1)}
\NormalTok{TobinsQ <-}\StringTok{ }\KeywordTok{lm}\NormalTok{(TobinsQ }\OperatorTok{~}\StringTok{ }\NormalTok{SustainabilityPayLink }\OperatorTok{+}\StringTok{ }\NormalTok{SustainableThemedCommitment }\OperatorTok{+}\StringTok{ }
\StringTok{    }\NormalTok{AuditScore }\OperatorTok{+}\StringTok{ }\NormalTok{CarbonProductivity }\OperatorTok{+}\StringTok{ }\NormalTok{WaterProductivity }\OperatorTok{+}\StringTok{ }
\StringTok{    }\NormalTok{WasteProductivity }\OperatorTok{+}\StringTok{ }\NormalTok{FirmSize }\OperatorTok{+}\StringTok{ }\NormalTok{Growth }\OperatorTok{+}\StringTok{ }\NormalTok{Leverage }\OperatorTok{+}\StringTok{ }\NormalTok{Industry, }
    \DataTypeTok{data =}\NormalTok{ ModelLag1)}
\CommentTok{# I calculate my cooks distance (i.e. D)}
\NormalTok{cooksdRoa <-}\StringTok{ }\KeywordTok{cooks.distance}\NormalTok{(Roa)}
\NormalTok{cooksdTobinsQ <-}\StringTok{ }\KeywordTok{cooks.distance}\NormalTok{(TobinsQ)}
\CommentTok{# I extract rows considered as influential (i.e.}
\CommentTok{# observations whose D > 4 * means) and I print them for}
\CommentTok{# the reader.}
\NormalTok{influentialRoa <-}\StringTok{ }\KeywordTok{as.numeric}\NormalTok{(}\KeywordTok{names}\NormalTok{(cooksdRoa)[(cooksdRoa }\OperatorTok{>}\StringTok{ }
\StringTok{    }\DecValTok{4} \OperatorTok{*}\StringTok{ }\KeywordTok{mean}\NormalTok{(cooksdRoa, }\DataTypeTok{na.rm =}\NormalTok{ T))])}
\NormalTok{influentialRoa}
\end{Highlighting}
\end{Shaded}

{[}1{]} 10 12 25 55 96 244 245 246 381 413 479 480 645 656 {[}15{]} 679
684 718 730 777 794 948 949 1106 1107 1108 1122 1123 1156 {[}29{]} 1171

\begin{Shaded}
\begin{Highlighting}[]
\NormalTok{influentialTobin <-}\StringTok{ }\KeywordTok{as.numeric}\NormalTok{(}\KeywordTok{names}\NormalTok{(cooksdTobinsQ)[(cooksdTobinsQ }\OperatorTok{>}\StringTok{ }
\StringTok{    }\DecValTok{4} \OperatorTok{*}\StringTok{ }\KeywordTok{mean}\NormalTok{(cooksdTobinsQ, }\DataTypeTok{na.rm =}\NormalTok{ T))])}
\NormalTok{influentialTobin}
\end{Highlighting}
\end{Shaded}

{[}1{]} 10 11 12 22 64 90 136 157 229 478 517 518 519 601 {[}15{]} 649
652 653 654 656 665 666 679 680 681 709 724 730 757 {[}29{]} 814 862 863
864 865 889 941 983 1043 1073 1074 1075 1085 1086 {[}43{]} 1107 1108
1122 1142

\begin{Shaded}
\begin{Highlighting}[]
\CommentTok{# I remove outliers and create two new dataframes that I}
\CommentTok{# write in my folders}
\NormalTok{TobinsQ_Db <-}\StringTok{ }\NormalTok{ModelLag1[}\OperatorTok{-}\KeywordTok{c}\NormalTok{(influentialTobin), ]}
\NormalTok{p <-}\StringTok{ "Analysis/DataBase/DataSynchronization/NoOutliersLag1/TobinsQ.csv"}
\KeywordTok{write.csv}\NormalTok{(TobinsQ_Db, }\DataTypeTok{file =}\NormalTok{ p)}
\NormalTok{p <-}\StringTok{ "Analysis/DataBase/DataSynchronization/NoOutliersLag1/Roa.csv"}
\NormalTok{Roa_Db <-}\StringTok{ }\NormalTok{ModelLag1[}\OperatorTok{-}\KeywordTok{c}\NormalTok{(influentialRoa), ]}
\KeywordTok{write.csv}\NormalTok{(Roa_Db, }\DataTypeTok{file =}\NormalTok{ p)}
\end{Highlighting}
\end{Shaded}

\begin{Shaded}
\begin{Highlighting}[]
\CommentTok{# I report influencial obervations on a graph}
\NormalTok{## TobinsQ plot cook's distance}
\KeywordTok{plot}\NormalTok{(cooksdTobinsQ, }\DataTypeTok{pch =} \StringTok{"*"}\NormalTok{, }\DataTypeTok{cex =} \DecValTok{2}\NormalTok{)}
\NormalTok{### add cutoff line}
\KeywordTok{abline}\NormalTok{(}\DataTypeTok{h =} \DecValTok{4} \OperatorTok{*}\StringTok{ }\KeywordTok{mean}\NormalTok{(cooksdTobinsQ, }\DataTypeTok{na.rm =}\NormalTok{ T), }\DataTypeTok{col =} \StringTok{"red"}\NormalTok{)}
\NormalTok{### add labels}
\KeywordTok{text}\NormalTok{(}\DataTypeTok{x =} \DecValTok{1}\OperatorTok{:}\KeywordTok{length}\NormalTok{(cooksdTobinsQ) }\OperatorTok{+}\StringTok{ }\DecValTok{1}\NormalTok{, }\DataTypeTok{y =}\NormalTok{ cooksdTobinsQ, }
    \DataTypeTok{labels =} \KeywordTok{ifelse}\NormalTok{(cooksdTobinsQ }\OperatorTok{>}\StringTok{ }\DecValTok{4} \OperatorTok{*}\StringTok{ }\KeywordTok{mean}\NormalTok{(cooksdTobinsQ, }
        \DataTypeTok{na.rm =}\NormalTok{ T), }\KeywordTok{names}\NormalTok{(cooksdTobinsQ), }\StringTok{""}\NormalTok{), }\DataTypeTok{col =} \StringTok{"red"}\NormalTok{)}
\end{Highlighting}
\end{Shaded}

\includegraphics{OutliersTreatment_files/figure-latex/unnamed-chunk-5-1.pdf}

\newpage

\begin{Shaded}
\begin{Highlighting}[]
\NormalTok{## Roa plot cook's distance}
\KeywordTok{plot}\NormalTok{(cooksdRoa, }\DataTypeTok{pch =} \StringTok{"*"}\NormalTok{, }\DataTypeTok{cex =} \DecValTok{2}\NormalTok{)}
\NormalTok{### add cutoff line}
\KeywordTok{abline}\NormalTok{(}\DataTypeTok{h =} \DecValTok{4} \OperatorTok{*}\StringTok{ }\KeywordTok{mean}\NormalTok{(cooksdRoa, }\DataTypeTok{na.rm =}\NormalTok{ T), }\DataTypeTok{col =} \StringTok{"red"}\NormalTok{)}
\NormalTok{### add labels}
\KeywordTok{text}\NormalTok{(}\DataTypeTok{x =} \DecValTok{1}\OperatorTok{:}\KeywordTok{length}\NormalTok{(cooksdRoa) }\OperatorTok{+}\StringTok{ }\DecValTok{1}\NormalTok{, }\DataTypeTok{y =}\NormalTok{ cooksdRoa, }\DataTypeTok{labels =} \KeywordTok{ifelse}\NormalTok{(cooksdRoa }\OperatorTok{>}\StringTok{ }
\StringTok{    }\DecValTok{4} \OperatorTok{*}\StringTok{ }\KeywordTok{mean}\NormalTok{(cooksdRoa, }\DataTypeTok{na.rm =}\NormalTok{ T), }\KeywordTok{names}\NormalTok{(cooksdRoa), }\StringTok{""}\NormalTok{), }
    \DataTypeTok{col =} \StringTok{"red"}\NormalTok{)}
\end{Highlighting}
\end{Shaded}

\includegraphics{OutliersTreatment_files/figure-latex/unnamed-chunk-6-1.pdf}


\end{document}
