\documentclass[]{article}
\usepackage{lmodern}
\usepackage{amssymb,amsmath}
\usepackage{ifxetex,ifluatex}
\usepackage{fixltx2e} % provides \textsubscript
\ifnum 0\ifxetex 1\fi\ifluatex 1\fi=0 % if pdftex
  \usepackage[T1]{fontenc}
  \usepackage[utf8]{inputenc}
\else % if luatex or xelatex
  \ifxetex
    \usepackage{mathspec}
  \else
    \usepackage{fontspec}
  \fi
  \defaultfontfeatures{Ligatures=TeX,Scale=MatchLowercase}
\fi
% use upquote if available, for straight quotes in verbatim environments
\IfFileExists{upquote.sty}{\usepackage{upquote}}{}
% use microtype if available
\IfFileExists{microtype.sty}{%
\usepackage{microtype}
\UseMicrotypeSet[protrusion]{basicmath} % disable protrusion for tt fonts
}{}
\usepackage[margin=1in]{geometry}
\usepackage{hyperref}
\hypersetup{unicode=true,
            pdfborder={0 0 0},
            breaklinks=true}
\urlstyle{same}  % don't use monospace font for urls
\usepackage{graphicx,grffile}
\makeatletter
\def\maxwidth{\ifdim\Gin@nat@width>\linewidth\linewidth\else\Gin@nat@width\fi}
\def\maxheight{\ifdim\Gin@nat@height>\textheight\textheight\else\Gin@nat@height\fi}
\makeatother
% Scale images if necessary, so that they will not overflow the page
% margins by default, and it is still possible to overwrite the defaults
% using explicit options in \includegraphics[width, height, ...]{}
\setkeys{Gin}{width=\maxwidth,height=\maxheight,keepaspectratio}
\IfFileExists{parskip.sty}{%
\usepackage{parskip}
}{% else
\setlength{\parindent}{0pt}
\setlength{\parskip}{6pt plus 2pt minus 1pt}
}
\setlength{\emergencystretch}{3em}  % prevent overfull lines
\providecommand{\tightlist}{%
  \setlength{\itemsep}{0pt}\setlength{\parskip}{0pt}}
\setcounter{secnumdepth}{0}
% Redefines (sub)paragraphs to behave more like sections
\ifx\paragraph\undefined\else
\let\oldparagraph\paragraph
\renewcommand{\paragraph}[1]{\oldparagraph{#1}\mbox{}}
\fi
\ifx\subparagraph\undefined\else
\let\oldsubparagraph\subparagraph
\renewcommand{\subparagraph}[1]{\oldsubparagraph{#1}\mbox{}}
\fi

%%% Use protect on footnotes to avoid problems with footnotes in titles
\let\rmarkdownfootnote\footnote%
\def\footnote{\protect\rmarkdownfootnote}

%%% Change title format to be more compact
\usepackage{titling}

% Create subtitle command for use in maketitle
\newcommand{\subtitle}[1]{
  \posttitle{
    \begin{center}\large#1\end{center}
    }
}

\setlength{\droptitle}{-2em}
  \title{}
  \pretitle{\vspace{\droptitle}}
  \posttitle{}
  \author{}
  \preauthor{}\postauthor{}
  \date{}
  \predate{}\postdate{}


\begin{document}

\section{Methodology}\label{methodology}

Here is my methodology\ldots{}

\subsection{Panel Data}\label{panel-data}

\subsubsection{Definition of panel data}\label{definition-of-panel-data}

Panel data, also called longitudinal data or cross-sectional time-series
data include observations on N cross section units (i.e., firms) over T
time-periods.

\subsubsection{Advantages of panel data
:}\label{advantages-of-panel-data}

As panel data analysis uses variation in both these dimensions, it is
considered to be one of the most efficient analytical methods for data
{[}@DimitriosAsteriou2006{]}. It usually contains more degrees of
freedom, less collinearity among the variables, more efficiency and more
sample variability than one-dimensional method (i.e.cross-sectional data
and time series data) giving a more accurate inference of the parameters
estimated in the model {[}@Hsiao2007, @HsiaoChapitrePanelData2014{]}.

\subsubsection{Fixed or random effect
model}\label{fixed-or-random-effect-model}

Panel data may have individual (group) effect, time effect, or both,
which are analyzed by fixed effect and/or random effect models. \emph{A
fixed effect model} examines if intercepts vary across group or time
period, whereas a \emph{random effect model} explores differences in
error variance components across individual or time period.
{[}@Park2011{]}.

\textbf{!! I need to test the fixed-random effect model of my database
before moving forward !!}

\begin{itemize}
\tightlist
\item
  @Ng2015 used the two-stage-least-square regressions to estimate its
  models.
\end{itemize}

** In case of presence of endogeneyity in an econometric model, OLS is
not capable of delivering consistent parameter estimates
{[}@Wooldridge2008{]}.**

Citation from {[}@Wooldridge2008{]} :

\begin{quote}
The general concept is that of the instrumental variables estimator; a
popular form of that estimator, often employed in the context of
endogeneity, is known as two-stage least squares (2SLS)
\end{quote}

\subsubsection{Endogeneity test}\label{endogeneity-test}

Even if panel data have a lot of advantages\ldots{}

Two issues involved in utilizing panel data, namely heterogeneity bias
and selectivity bias {[}@HsiaoChapitrePanelData2014{]}.

Citation from @HsiaoChapitrePanelData2014:

\begin{quote}
It is only by taking proper account of selectivity and heterogeneity
biases in the panel data that one can have confidence in the results
obtained.
\end{quote}

@Dangsearchrobustmethods2015 examine which methods are appropriate for
estimating dynamic panel data models in empirical corporate
finance,especially in short panels of company data, in the likely
presence of (1) unobserved heterogeneity and endogeneity, (2) residual
serial correlation, or (3) fractional dependent variables. The
bias-corrected fixed-effects estimators, based on an analytical,
bootstrap, or indirect inference approach, are found to be the most
appropriate and robust methods.

But @MiroshnychenkoGreenpracticesfinancial2017 used the OLS regressions
in micro panel using the Huber-White sand which estimator, to account
for the heteroscedasticity problem\ldots{} \textbf{Which method should I
use?}

Hausmann test to test the random effects model for both dependent
variables?

\newpage

\subsection{Econometric Model}\label{econometric-model}

The first hypothesis will be tested with T-tests on the impact of each
green initiative on green performance.

Hypotheses two and three will be tested by regression analysis using the
\emph{plm} package. Econometric models are based on @Delmas2015 and
@MiroshnychenkoGreenpracticesfinancial2017 and started from the general
form:

\begin{equation}
Y_{t+1}=\beta_{0} + \beta_{1} (X_{it}) + + \beta_{2} (C_{it}) + \varepsilon_{it}
\label{GeneralForm}
\end{equation}

where \(Y_{t+1}\) is the financial performance of firm \(i\) in year
\(t+1\), \(\beta\) is the vector of estimated regression coefficients
for each of the explanatory variables \(X_{it}\), \(C_{it}\) is a vector
of control variables , \(\varepsilon_{it}\) is the error term.

More precisely I will regress six models :

\textbf{Model 1 :} Green Initiatives on Tobin's Q

\begin{equation}
TobinsQ_{it+1} = \beta_{0} + \beta_{1} (SP_{it}) + \beta_{2} (ST_{it}) + \beta_{3} (AS_{it}) + \beta_{4} (C_{it}) + \varepsilon_{it}
\label{M1}
\end{equation}

\textbf{Model 2 :} Green Initiatives on ROA

\begin{equation}
ROA_{it+1} = \beta_{0} + \beta_{1} (SP_{it}) + \beta_{2} (ST_{it}) + \beta_{3} (AS_{it}) + \beta_{4} (C_{it}) + \varepsilon_{it}
\label{M2}
\end{equation}

\textbf{Model 3 :} Green Performance on Tobin's Q

\begin{equation}
TobinsQ_{it+1} = \beta_{0} + \beta_{1} (EP_{it}) + \beta_{2} (CP_{it}) + \beta_{3} (WatP_{it}) + \beta_{4} (WasP_{it}) + \beta_{5} (C_{it}) + \varepsilon_{it}
\label{M3}
\end{equation}

\textbf{Model 4 :} Green Performance on ROA

\begin{equation}
ROA_{it+1} = \beta_{0} + \beta_{1} (EP_{it}) + \beta_{2} (CP_{it}) + \beta_{3} (WatP_{it}) + \beta_{4} (WasP_{it}) + \beta_{5} (C_{it}) + \varepsilon_{it}
\label{M4}
\end{equation}

\textbf{Model 5 :} Both Green Performance and Green Initiative on
Tobin's Q

\begin{equation}
TobinsQ_{it+1} = \beta_{0} + \beta_{1} (EP_{it}) + \beta_{2} (CP_{it}) + \beta_{3} (WatP_{it}) + \beta_{4} (WasP_{it})  + \beta_{5} (SP_{it}) + \beta_{6} (ST_{it}) + \beta_{7} (AS_{it})+ (C_{it}) + \varepsilon_{it}
\label{M5}
\end{equation}

\textbf{Model 6 :} Both Green Performance and Green Initiative on ROA

\begin{equation}
ROA_{it+1} = \beta_{0} + \beta_{1} (EP_{it}) + \beta_{2} (CP_{it}) + \beta_{3} (WatP_{it}) + \beta_{4} (WasP_{it})  + \beta_{5} (SP_{it}) + \beta_{6} (ST_{it}) + \beta_{7} (AS_{it})+ (C_{it}) + \varepsilon_{it}
\label{M6}
\end{equation}

where :

\begin{itemize}
\tightlist
\item
  \(TobinsQ_{it+1}\) = a proxy for a firm's financial performance
\item
  \(ROA_{it+1}\) = a proxy for a firm's financial performance
\item
  \(EP_{it}\) = a proxy for a firm's energy productivity
\item
  \(CP_{it}\) = a proxy for a firm's carbon productivity
\item
  \(WatP_{it}\) = a proxy for a firm's water productivity
\item
  \(WasP_{it}\) = a proxy for a firm's waste productivity
\item
  \(SP_{it}\) = a proxy for a firm's sustainability pay link
\item
  \(ST_{it}\) = a proxy for a firm's sustainability themed commitment
\item
  \(EP_{it}\) = a proxy for a firm's audit score
\item
  \(C_{it}\) = a vector of control variables that include financial
  leverage, firm size, net margin and industry sector
\item
  \(\varepsilon_{it}\) = the error term
\end{itemize}

\newpage

\subsection{Panel Data Tests}\label{panel-data-tests}

This section will not be in the final document but in appendix. It is
only to report the result of the bunch of tests I carried out in order
to define which panel data methotolodies I will use for each one of my 6
models.

@Croissant2008a and @Torres-Reyna2010 really helped me.

Here are the tests :

\begin{enumerate}
\def\labelenumi{\arabic{enumi}.}
\tightlist
\item
  Test of poolability
\item
  Hausmann Test to determine the fixed or random effect
\item
  Test for time fixed effect
\item
  Test for cross-sectional dependence
\item
  Test for serial correlation
\item
  Test for stationarity
\item
  Test for heteroskedasticity
\end{enumerate}

The table \ref{TestSummary} summaries the result of each test for each
model. You can find details below.

Regarding the poolability test I have an issue with my code that I still
need to solve. This is why it is written \emph{NA} in the table
\ref{TestSummary}.

\begin{table}[h]
\centering
\begin{tabular}{rrrrrrr}
\hline
 & {Model 1} & {Model 2} & {Model 3} & {Model 4} & {Model 5} & {Model 6} \\ 
\hline
{Poolability} & NA & NA & NA & NA & NA & NA \\
{Hausmann} & Fixed & Fixed & Fixed & Fixed & Fixed & Fixed \\
{Time Fixed Effect} & No & Yes & No & Yes & No & Yes \\
{Cross Sectional Dependence} & Yes & Yes & Yes & No & No & No \\
{Serial Correlation} & Yes & Yes & Yes & Yes & Yes & Yes \\
{Stationarity} & None & None & None & None & None & None \\
{Heteroskedasticity} & Yes & Yes & Yes & Yes & Yes & Yes \\ 
\hline
\end{tabular}
\caption{Test Summary}
\label{TestSummary}
\end{table}

\newpage

\subsubsection{Test of poolability}\label{test-of-poolability}

Citation from {[}@Croissant2008{]} :

\begin{quote}
\emph{Pooltest tests the hypothesis that the same coefficients apply to
each individual. It is a standard F test, based on the comparison of a
model obtained for the full sample and a model based on the estimation
of an equation for each individual. The first argument of pooltest is a
plm object. The second argument is a pvcm object obtained with
model=within. If the first argument is a pooling model, the test applies
to all the coefficients (including the intercepts), if it is a within
model, different intercepts are assumed.}
\end{quote}

To carry out the of poolabiloty I have used the \emph{pooltest}
function. The null hypothesis of poolability assumes homogeneous slope
coefficients.

When running my code I got this error : Error in FUN(X{[}{[}i{]}{]},
\ldots{}) : insufficient number of observations

I still need to understand the origin of this error.

\subsubsection{Hausmann Test to determine the fixed or random
effect}\label{hausmann-test-to-determine-the-fixed-or-random-effect}

Citation from @Torres-Reyna2010 :

\begin{quote}
\emph{To decide between fixed or random effects you can run a Hausman
test where the null hypothesis is that the preferred model is random
effects vs.~the alternative the fixed effects (see Green, 2008, chapter
9). It basically tests whether the unique errors (ui) are correlated
with the regressors, the null hypothesis is they are not.}
\end{quote}

The \autoref{Hausman} summarizes results of the Hausman Test of each
model. I hae used the \emph{phtest} function to carry out this test. We
can observe that all p-values are \textless{} 0.05 meaning that HO is
not verified and my models are caraterized by a fixed effect.

\begin{table}[h] \centering 
  \caption{Hausman Test PValue} 
  \label{Hausman} 
\begin{tabular}{@{\extracolsep{5pt}} cc} 
\\[-1.8ex]\hline 
\hline \\[-1.8ex] 
Model & P-Value \\ 
\hline \\[-1.8ex] 
Model 1 & 9.02236417939369e-10 \\ 
Model 2 & 0.00302619725407582 \\ 
Model 3 & 1.37656596228424e-07 \\ 
Model 4 & 3.57999207647531e-05 \\ 
Model 5 & 4.64415816764099e-07 \\ 
Model 6 & 6.0042809986031e-06 \\ 
\hline \\[-1.8ex] 
\end{tabular} 
\end{table}

\newpage

\newpage

\subsubsection{Test for time fixed
effect}\label{test-for-time-fixed-effect}

The \autoref{pFtest} summarizes results of the test for each model. I
have used the \emph{pFtest} function to carry out this test.

P-Value is \textgreater{} 0.05 for model 1, model 3 and model 5 meaning
that null hypothesis is verified and that there is not a significant
time-fixed effect. However for model 2,model 4 and model 6 P-Value is
\textless{} 0.05 meaning that null hypothesis is rejected and that there
is a significant time-fixed effect.

\textbf{Does this mean that for model 2,4 and 6 I have to add the time
fixed effect in my model?}

\begin{table}[h] \centering 
  \caption{Fixed Time Effect Test PValue} 
  \label{pFtest} 
\begin{tabular}{@{\extracolsep{5pt}} ccc} 
\\[-1.8ex]\hline 
\hline \\[-1.8ex] 
 & Model & P-Value \\ 
\hline \\[-1.8ex] 
F & Model 1 & 0.278369414243633 \\ 
F & Model 2 & 1.16802571710936e-07 \\ 
F & Model 3 & 0.406328079743586 \\ 
F & Model 4 & 2.02792272227493e-06 \\ 
F & Model 5 & 0.533510997462999 \\ 
F & Model 6 & 9.11828173366536e-07 \\ 
\hline \\[-1.8ex] 
\end{tabular} 
\end{table}

\newpage

\subsubsection{Test for cross-sectional
dependence}\label{test-for-cross-sectional-dependence}

Citation from @Torres-Reyna2010 :

\begin{quote}
\emph{According to Baltagi, cross-sectional dependence is a problem in
macro panels with long time series. This is not much of a problem in
micro panels (few years and large number of cases). The null hypothesis
in the B -P/LM and Pasaran CD tests of independence is that residuals
across entities are not correlated. B- P/LM and Pasaran CD
(cross-sectional dependence) tests are used to test whether the
residuals are correlated across entities. Cross-sectional dependence can
lead to bias in tests results (also called contemporaneous
correlation).}
\end{quote}

I have used the \emph{pcdtest} function to carry out this test. The
\autoref{pcd} show results of the test for cross-sectional dependence.
We can observe that I have cross-sectional dependence in my model 1,2
and 3. However for model 4, 5 and 6, the P-Value is superior to 0.05
meaning that HO is verified and these models do not have cross-sectional
dependence.

\begin{table}[h] \centering 
  \caption{Cross-sectional dependence's test - PValue} 
  \label{pcd} 
\begin{tabular}{@{\extracolsep{5pt}} ccc} 
\\[-1.8ex]\hline 
\hline \\[-1.8ex] 
Model & Method & P-Value \\ 
\hline \\[-1.8ex] 
Model 1 & cd & 5.15547149343488e-07 \\ 
Model 2 & cd & 1.29203892499336e-13 \\ 
Model 3 & cd & 0.000580300093082629 \\ 
Model 4 & cd & 0.0778453698933712 \\ 
Model 5 & cd & 0.155469112754022 \\ 
Model 6 & cd & 0.11461956717495 \\ 
\hline \\[-1.8ex] 
\multicolumn{3}{l}{Note :'cd' stands for Pesaran's CD Statistic} \\ 
\end{tabular} 
\end{table}

\newpage

\subsubsection{Test for serial
correlation}\label{test-for-serial-correlation}

I used the Wooldridge's test for serial correlation in FE panels with
the \emph{pwartest} function to test the serial correlation of my
models. According to @Croissant2008 this test is applicable to any fixed
effect panel model, and in particular to short panels with small T and
large n, which is my case. The null hypothese is that there is no serial
correlation in the model. According to the P\_Value of my models, I can
conclude that I have serial correlation in all models.

\begin{table}[h] \centering 
  \caption{Wooldridge's test - PValue} 
  \label{pwartest} 
\begin{tabular}{@{\extracolsep{5pt}} ccc} 
\\[-1.8ex]\hline 
\hline \\[-1.8ex] 
 & Model & P-Value \\ 
\hline \\[-1.8ex] 
F & Model 1 & 4.0693722372745e-06 \\ 
F & Model 2 & 0.00742990553057738 \\ 
F & Model 3 & 5.1340385894311e-06 \\ 
F & Model 4 & 0.0216883877586851 \\ 
F & Model 5 & 5.2306368288514e-06 \\ 
F & Model 6 & 0.0314278145382219 \\ 
\hline \\[-1.8ex] 
\end{tabular} 
\end{table}

\newpage

\subsubsection{Test for stationarity}\label{test-for-stationarity}

The Dickey-Fuller test to check for stochastic trends with the
\emph{adf.test} function. The null hypothesis is that the series has a
unit root (i.e.~non-stationary). In my case HO is rejected for both
databases meaning that they do not have stationarity.

\begin{table}[h] \centering 
  \caption{Dickey-Fuller test - PValue} 
  \label{Stationarity} 
\begin{tabular}{@{\extracolsep{5pt}} cc} 
\\[-1.8ex]\hline 
\hline \\[-1.8ex] 
Database & P\_Value \\ 
\hline \\[-1.8ex] 
Roa & 0.01 \\ 
TobinsQ & 0.01 \\ 
\hline \\[-1.8ex] 
\end{tabular} 
\end{table}

\newpage

\subsubsection{Test for
heteroskedasticity}\label{test-for-heteroskedasticity}

I have used the \emph{Bptest} function to test the presence of
heteroskedasticity of my model. The \autoref{Hetero} summarizes the
p-value of each model. I find strange that all p-value equals zero. By
meaning a p-value cannot be null, right? \textbf{What do you think?}

\begin{table}[h] \centering 
  \caption{Heteroskedasticity Test - PValue} 
  \label{Hetero} 
\begin{tabular}{@{\extracolsep{5pt}} ccc} 
\\[-1.8ex]\hline 
\hline \\[-1.8ex] 
 & Model & P\_Value \\ 
\hline \\[-1.8ex] 
BP & Model 1 & 0 \\ 
BP & Model 2 & 0 \\ 
BP & Model 3 & 0 \\ 
BP & Model 4 & 0 \\ 
BP & Model 5 & 0 \\ 
BP & Model 6 & 0 \\ 
\hline \\[-1.8ex] 
\end{tabular} 
\end{table}

Starting from the premise that I have heteroskedasiticy I will compute
the \textbf{sandwich estimators} of my models. The \autoref{Sand}
summarizes the sandwich estimators for each model. \textbf{What should I
do with that?}

\begin{table}[h] \centering 
  \caption{Sandwich Estimators} 
  \label{Sand} 
\begin{tabular}{@{\extracolsep{1pt}}lcccccc} 
\\[-1.8ex]\hline 
\hline \\[-1.8ex] 
 & \multicolumn{6}{c}{\textit{Dependent variable:}} \\ 
\cline{2-7} 
\\[-1.8ex] & \multicolumn{6}{c}{ } \\ 
\\[-1.8ex] & (1) & (2) & (3) & (4) & (5) & (6)\\ 
\hline \\[-1.8ex] 
 SustainabilityPayLink & 0.778$^{*}$ & $-$0.063 &  &  & 0.536 & $-$0.048 \\ 
  & (0.461) & (0.055) &  &  & (0.581) & (0.057) \\ 
  & & & & & & \\ 
 SustainableThemedCommitment & 3.473 & 0.332$^{**}$ &  &  & 2.889 & 0.384$^{**}$ \\ 
  & (2.516) & (0.153) &  &  & (2.613) & (0.149) \\ 
  & & & & & & \\ 
 AuditScore & 0.452 & 0.006 &  &  & 0.083 & 0.022 \\ 
  & (0.931) & (0.081) &  &  & (0.911) & (0.087) \\ 
  & & & & & & \\ 
 EnergyProductivity &  &  & 0.077 & 0.019 & 0.094 & 0.018 \\ 
  &  &  & (0.150) & (0.017) & (0.150) & (0.016) \\ 
  & & & & & & \\ 
 CarbonProductivity &  &  & $-$0.078 & $-$0.038$^{**}$ & $-$0.051 & $-$0.039$^{**}$ \\ 
  &  &  & (0.172) & (0.019) & (0.179) & (0.019) \\ 
  & & & & & & \\ 
 WaterProductivity &  &  & $-$0.091 & 0.036$^{***}$ & $-$0.092 & 0.037$^{***}$ \\ 
  &  &  & (0.130) & (0.014) & (0.131) & (0.014) \\ 
  & & & & & & \\ 
 WasteProductivity &  &  & $-$0.206$^{*}$ & 0.0004 & $-$0.182 & 0.003 \\ 
  &  &  & (0.109) & (0.009) & (0.111) & (0.010) \\ 
  & & & & & & \\ 
 Leverage & $-$0.00004 & $-$0.00003 & $-$0.00005 & $-$0.00003 & $-$0.00005 & $-$0.00003 \\ 
  & (0.0002) & (0.00003) & (0.0002) & (0.00003) & (0.0002) & (0.00003) \\ 
  & & & & & & \\ 
 NetMargin & $-$0.007 & 0.053 & $-$0.010 & 0.052 & $-$0.008 & 0.052 \\ 
  & (0.039) & (0.064) & (0.039) & (0.064) & (0.040) & (0.064) \\ 
  & & & & & & \\ 
 FirmSize & $-$0.318 & $-$0.001 & $-$0.313 & 0.001 & $-$0.320 & 0.0003 \\ 
  & (0.432) & (0.021) & (0.429) & (0.021) & (0.432) & (0.021) \\ 
  & & & & & & \\ 
\hline \\[-1.8ex] 
\hline 
\hline \\[-1.8ex] 
\textit{Note:}  & \multicolumn{6}{r}{$^{*}$p$<$0.1; $^{**}$p$<$0.05; $^{***}$p$<$0.01} \\ 
\end{tabular} 
\end{table}

See @MiroshnychenkoGreenpracticesfinancial2017 and @Stock2008

\emph{If hetersokedaticity is detected you can use the sandwich
estimaror} {[}@Torres-Reyna2010{]}

vcovHC is a function for estimating a robust covariance matrix of
parameters for a fixed effects or random effects panel model according
to the White method (White 1980, 1984; Arellano 1987). The --vcovHC–
function estimates three heteroskedasticity-consistent covariance
estimators:

\begin{itemize}
\item
  ``white1'' - for general heteroskedasticity but no serial correlation.
  Recommended for random effects.
\item
  ``white2'' - is ``white1'' restricted to a common variance within
  groups. Recommended for random effects.
\item
  ``arellano'' - both heteroskedasticity and serial correlation.
  Recommended for fixed effects.
\end{itemize}

The following options apply*:

\begin{itemize}
\tightlist
\item
  HC0 - heteroskedasticity consistent. The default.
\item
  HC1,HC2, HC3 – Recommended for small samples. HC3 gives less weight
  to influential observations.
\item
  HC4 - small samples with influential observations
\item
  HAC - heteroskedasticity and autocorrelation consistent (type ?vcovHAC
  for more details)
\end{itemize}

\subsection{Sensitivity Analysis}\label{sensitivity-analysis}


\end{document}
