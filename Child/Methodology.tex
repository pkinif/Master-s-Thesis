\documentclass[]{article}
\usepackage{lmodern}
\usepackage{amssymb,amsmath}
\usepackage{ifxetex,ifluatex}
\usepackage{fixltx2e} % provides \textsubscript
\ifnum 0\ifxetex 1\fi\ifluatex 1\fi=0 % if pdftex
  \usepackage[T1]{fontenc}
  \usepackage[utf8]{inputenc}
\else % if luatex or xelatex
  \ifxetex
    \usepackage{mathspec}
  \else
    \usepackage{fontspec}
  \fi
  \defaultfontfeatures{Ligatures=TeX,Scale=MatchLowercase}
\fi
% use upquote if available, for straight quotes in verbatim environments
\IfFileExists{upquote.sty}{\usepackage{upquote}}{}
% use microtype if available
\IfFileExists{microtype.sty}{%
\usepackage{microtype}
\UseMicrotypeSet[protrusion]{basicmath} % disable protrusion for tt fonts
}{}
\usepackage[margin=1in]{geometry}
\usepackage{hyperref}
\hypersetup{unicode=true,
            pdfborder={0 0 0},
            breaklinks=true}
\urlstyle{same}  % don't use monospace font for urls
\usepackage{graphicx,grffile}
\makeatletter
\def\maxwidth{\ifdim\Gin@nat@width>\linewidth\linewidth\else\Gin@nat@width\fi}
\def\maxheight{\ifdim\Gin@nat@height>\textheight\textheight\else\Gin@nat@height\fi}
\makeatother
% Scale images if necessary, so that they will not overflow the page
% margins by default, and it is still possible to overwrite the defaults
% using explicit options in \includegraphics[width, height, ...]{}
\setkeys{Gin}{width=\maxwidth,height=\maxheight,keepaspectratio}
\IfFileExists{parskip.sty}{%
\usepackage{parskip}
}{% else
\setlength{\parindent}{0pt}
\setlength{\parskip}{6pt plus 2pt minus 1pt}
}
\setlength{\emergencystretch}{3em}  % prevent overfull lines
\providecommand{\tightlist}{%
  \setlength{\itemsep}{0pt}\setlength{\parskip}{0pt}}
\setcounter{secnumdepth}{0}
% Redefines (sub)paragraphs to behave more like sections
\ifx\paragraph\undefined\else
\let\oldparagraph\paragraph
\renewcommand{\paragraph}[1]{\oldparagraph{#1}\mbox{}}
\fi
\ifx\subparagraph\undefined\else
\let\oldsubparagraph\subparagraph
\renewcommand{\subparagraph}[1]{\oldsubparagraph{#1}\mbox{}}
\fi

%%% Use protect on footnotes to avoid problems with footnotes in titles
\let\rmarkdownfootnote\footnote%
\def\footnote{\protect\rmarkdownfootnote}

%%% Change title format to be more compact
\usepackage{titling}

% Create subtitle command for use in maketitle
\newcommand{\subtitle}[1]{
  \posttitle{
    \begin{center}\large#1\end{center}
    }
}

\setlength{\droptitle}{-2em}
  \title{}
  \pretitle{\vspace{\droptitle}}
  \posttitle{}
  \author{}
  \preauthor{}\postauthor{}
  \date{}
  \predate{}\postdate{}


\begin{document}

\section{Methodology}\label{methodology}

Here is my methodology\ldots{}

\subsection{Panel Data}\label{panel-data}

\subsubsection{Definition of panel data}\label{definition-of-panel-data}

Panel data, also called longitudinal data or cross-sectional time-series
data include observations on N cross section units (i.e., firms) over T
time-periods.

\subsubsection{Advantages of panel data
:}\label{advantages-of-panel-data}

As panel data analysis uses variation in both these dimensions, it is
considered to be one of the most efficient analytical methods for data
{[}@DimitriosAsteriou2006{]}. It usually contains more degrees of
freedom, less collinearity among the variables, more efficiency and more
sample variability than one-dimensional method (i.e.cross-sectional data
and time series data) giving a more accurate inference of the parameters
estimated in the model {[}@Hsiao2007, @HsiaoChapitrePanelData2014{]}.

\subsubsection{Fixed or random effect
model}\label{fixed-or-random-effect-model}

Panel data may have individual (group) effect, time effect, or both,
which are analyzed by fixed effect and/or random effect models. \emph{A
fixed effect model} examines if intercepts vary across group or time
period, whereas a \emph{random effect model} explores differences in
error variance components across individual or time period.
{[}@Park2011{]}.

\textbf{!! I need to test the fixed-random effect model of my database
before moving forward !!}

\begin{itemize}
\tightlist
\item
  @Ng2015 used the two-stage-least-square regressions to estimate its
  models.
\end{itemize}

** In case of presence of endogeneyity in an econometric model, OLS is
not capable of delivering consistent parameter estimates
{[}@Wooldridge2008{]}.**

Citation from {[}@Wooldridge2008{]} :

\begin{quote}
The general concept is that of the instrumental variables estimator; a
popular form of that estimator, often employed in the context of
endogeneity, is known as two-stage least squares (2SLS)
\end{quote}

\subsubsection{Endogeneity test}\label{endogeneity-test}

Even if panel data have a lot of advantages\ldots{}

Two issues involved in utilizing panel data, namely heterogeneity bias
and selectivity bias {[}@HsiaoChapitrePanelData2014{]}.

Citation from @HsiaoChapitrePanelData2014:

\begin{quote}
It is only by taking proper account of selectivity and heterogeneity
biases in the panel data that one can have confidence in the results
obtained.
\end{quote}

@Dangsearchrobustmethods2015 examine which methods are appropriate for
estimating dynamic panel data models in empirical corporate
finance,especially in short panels of company data, in the likely
presence of (1) unobserved heterogeneity and endogeneity, (2) residual
serial correlation, or (3) fractional dependent variables. The
bias-corrected fixed-effects estimators, based on an analytical,
bootstrap, or indirect inference approach, are found to be the most
appropriate and robust methods.

But @MiroshnychenkoGreenpracticesfinancial2017 used the OLS regressions
in micro panel using the Huber-White sand wich estimator, to account for
the heteroscedasticity problem\ldots{} \textbf{Which method should I
use?}

Hausmann test to test the random effects model for both dependant
variables?

\subsection{Econometric Model}\label{econometric-model}

The first hypothesis will be tested with T-tests on the impact of each
green initiative on green performance.

Both Hypotheses two and three will be tested by regression analysis.
Econometric models are based on @Delmas2015 and
@MiroshnychenkoGreenpracticesfinancial2017 and started from the general
form:

\begin{equation}
Y_{t+1}=\beta_{0} + \beta_{1} (X_{it}) + + \beta_{2} (C_{it}) + \varepsilon_{it}
\label{GeneralForm}
\end{equation}

where \(Y_{t+1}\) is the financial performance of firm \(i\) in year
\(t+1\), \(\beta\) is the vector of estimated regression coefficients
for each of the explanatory variables \(X_{it}\), \(C_{it}\) is a vector
of control variables that include financial leverage, firm size and
industry sector, \(\varepsilon_{it}\)is the error term.

More precisely I will test six models :

\textbf{Model 1 :} Green Initiatives on Tobin's Q

\begin{equation}
TobinsQ_{it+1}=\beta_{0} + \beta_{1} (SP_{it}) + \beta_{2} (ST_{it}) + \beta_{3} (AS_{it}) + \beta_{9} (C_{it}) + \varepsilon_{it}
\label{M1}
\end{equation}

\textbf{Model 2 :} Green Initiatives on ROA

\begin{equation}
ROA_{it+1}=\beta_{0} + \beta_{1} (SP_{it}) + \beta_{2} (ST_{it}) + \beta_{3} (AS_{it}) + \beta_{9} (C_{it}) + \varepsilon_{it}
\label{M2}
\end{equation}

\textbf{Model 3 :} Green Performance on Tobin's Q

\begin{equation}
TobinsQ_{it+1}=\beta_{0} + \beta_{1} (EP_{it}) + \beta_{2} (CP_{it}) + \beta_{3} (WatP_{it}) + \beta_{4} (WasP_{it}) + \beta_{5} (GP_{it}) + \beta_{9} (C_{it}) + \varepsilon_{it}
\label{M3}
\end{equation}

\textbf{Model 4 :} Green Performance on ROA

\begin{equation}
ROA_{it+1}=\beta_{0} + \beta_{1} (EP_{it}) + \beta_{2} (CP_{it}) + \beta_{3} (WatP_{it}) + \beta_{4} (WasP_{it}) + \beta_{5} (GP_{it}) + \beta_{9} (C_{it}) + \varepsilon_{it}
\label{M4}
\end{equation}

\textbf{Model 5 :} Both Green Performance and Green Initiative on
Tobin's Q

\begin{equation}
TobinsQ_{it+1}=\beta_{0} + \beta_{1} (EP_{it}) + \beta_{2} (CP_{it}) + \beta_{3} (WatP_{it}) + \beta_{4} (WasP_{it}) + \beta_{5} (GP_{it}) + \beta_{6} (SP_{it}) + \beta_{7} (ST_{it}) + \beta_{8} (AS_{it}) + \beta_{9} (C_{it}) + \varepsilon_{it}
\label{M5}
\end{equation}

\textbf{Model 6 :} Both Green Performance and Green Initiative on ROA

\begin{equation}
ROA_{it+1}=\beta_{0} + \beta_{1} (EP_{it}) + \beta_{2} (CP_{it}) + \beta_{3} (WatP_{it}) + \beta_{4} (WasP_{it}) + \beta_{5} (GP_{it}) + \beta_{6} (SP_{it}) + \beta_{7} (ST_{it}) + \beta_{8} (AS_{it})+ (C_{it}) + \varepsilon_{it}
\label{M6}
\end{equation}

where :

\begin{itemize}
\tightlist
\item
  \(TobinsQ_{it+1}\) = a proxy for a firm's financial performance
\item
  \(ROA_{it+1}\) = a proxy for a firm's financial performance
\item
  \(EP_{it}\) = a proxy for a firm's energy productivity
\item
  \(CP_{it}\) = a proxy for a firm's carbon productivity
\item
  \(WatP_{it}\) = a proxy for a firm's water productivity
\item
  \(WasP_{it}\) = a proxy for a firm's waste productivity
\item
  \(GP_{it}\) = a proxy for a firm's green reputation
\item
  \(SP_{it}\) = a proxy for a firm's sustainability pay link
\item
  \(ST_{it}\) = a proxy for a firm's sustainability themed commitment
\item
  \(EP_{it}\) = a proxy for a firm's audit score
\item
  \(C_{it}\) = a vector of control variables that include financial
  leverage, firm size and industry sector
\item
  \(\varepsilon_{it}\) = the error term
\end{itemize}

Dans l'\autoref{M1}, blabla ou dans l'équation \ref{M1}

\subsection{Sensitivity Analysis}\label{sensitivity-analysis}


\end{document}
