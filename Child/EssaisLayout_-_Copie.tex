\documentclass[12pt,]{article}
\usepackage{lmodern}
\usepackage{amssymb,amsmath}
\usepackage{ifxetex,ifluatex}
\usepackage{fixltx2e} % provides \textsubscript
\ifnum 0\ifxetex 1\fi\ifluatex 1\fi=0 % if pdftex
  \usepackage[T1]{fontenc}
  \usepackage[utf8]{inputenc}
\else % if luatex or xelatex
  \ifxetex
    \usepackage{mathspec}
  \else
    \usepackage{fontspec}
  \fi
  \defaultfontfeatures{Ligatures=TeX,Scale=MatchLowercase}
\fi
% use upquote if available, for straight quotes in verbatim environments
\IfFileExists{upquote.sty}{\usepackage{upquote}}{}
% use microtype if available
\IfFileExists{microtype.sty}{%
\usepackage{microtype}
\UseMicrotypeSet[protrusion]{basicmath} % disable protrusion for tt fonts
}{}
\usepackage[margin = 1.2in]{geometry}
\usepackage{hyperref}
\PassOptionsToPackage{usenames,dvipsnames}{color} % color is loaded by hyperref
\hypersetup{unicode=true,
            colorlinks=true,
            linkcolor=black,
            citecolor=blue,
            urlcolor=black,
            breaklinks=true}
\urlstyle{same}  % don't use monospace font for urls
\usepackage{graphicx,grffile}
\makeatletter
\def\maxwidth{\ifdim\Gin@nat@width>\linewidth\linewidth\else\Gin@nat@width\fi}
\def\maxheight{\ifdim\Gin@nat@height>\textheight\textheight\else\Gin@nat@height\fi}
\makeatother
% Scale images if necessary, so that they will not overflow the page
% margins by default, and it is still possible to overwrite the defaults
% using explicit options in \includegraphics[width, height, ...]{}
\setkeys{Gin}{width=\maxwidth,height=\maxheight,keepaspectratio}
\IfFileExists{parskip.sty}{%
\usepackage{parskip}
}{% else
\setlength{\parindent}{0pt}
\setlength{\parskip}{6pt plus 2pt minus 1pt}
}
\setlength{\emergencystretch}{3em}  % prevent overfull lines
\providecommand{\tightlist}{%
  \setlength{\itemsep}{0pt}\setlength{\parskip}{0pt}}
\setcounter{secnumdepth}{5}
% Redefines (sub)paragraphs to behave more like sections
\ifx\paragraph\undefined\else
\let\oldparagraph\paragraph
\renewcommand{\paragraph}[1]{\oldparagraph{#1}\mbox{}}
\fi
\ifx\subparagraph\undefined\else
\let\oldsubparagraph\subparagraph
\renewcommand{\subparagraph}[1]{\oldsubparagraph{#1}\mbox{}}
\fi

%%% Use protect on footnotes to avoid problems with footnotes in titles
\let\rmarkdownfootnote\footnote%
\def\footnote{\protect\rmarkdownfootnote}

%%% Change title format to be more compact
\usepackage{titling}

% Create subtitle command for use in maketitle
\newcommand{\subtitle}[1]{
  \posttitle{
    \begin{center}\large#1\end{center}
    }
}

\setlength{\droptitle}{-2em}
  \title{}
  \pretitle{\vspace{\droptitle}}
  \posttitle{}
  \author{}
  \preauthor{}\postauthor{}
  \date{}
  \predate{}\postdate{}

\usepackage{placeins}
\usepackage{fancyhdr}
\usepackage{setspace}
\usepackage{chngcntr}
\onehalfspacing
\counterwithin{figure}{section}
\counterwithin{table}{section}

\begin{document}

\pagenumbering{gobble}

\begin{centering}

\vspace{3 cm}

\Huge

{\bf Here I have to write the title of my thesis}

\vspace{3 cm}

\Large
KINIF Pierrick

\vspace{3 cm}


\normalsize
Submitted in partial fulfilment of the requirements of the Master's Degree of Business Management and Administration, Finance Specialisation



May 2018

\vspace{3 cm}

\normalsize

Faculty of Economics, Social Sciences and Business Administration

\normalsize
University of Namur

\vspace{2 cm}


![caption](\figures\UNamur.png)


\end{centering}

\newpage

\pagenumbering{roman}

\section*{Abstract}\label{abstract}
\addcontentsline{toc}{section}{Abstract}

This is an abstract

\newpage

\section*{Aknowledgment}\label{aknowledgment}
\addcontentsline{toc}{section}{Aknowledgment}

I would like to thank some of you \ldots{}

\newpage

\setcounter{tocdepth}{2} \tableofcontents

\newpage

\addcontentsline{toc}{section}{List of Figures}

\listoftables

\newpage

\addcontentsline{toc}{section}{List of Tables}

\listoffigures

\newpage

\pagenumbering{arabic}

\section*{Introduction}\label{introduction}
\addcontentsline{toc}{section}{Introduction}

Over the past decades, humanity is progressively becoming aware of the
finiteness of earth's resources and its impact on the current global
warming. On the one hand, Houghton and Change (1996) anticipated in
their first report an average global warming between +1° and +3.5° C
until 2100 relative to the temperature of 1990. They also warned that an
increase of temperatures superior to +2° C could have some harsh
climatic repercussions. On the other hand the Kyoto Protocol had been
written in 1997, enforced in 2005 and led to the first Global Agreement
on global warming during the Paris Conference in 2015. Those different
solutions implemented over the past decades did not have any significant
impacts on the fight against global warming. Greenhouse Gas Emissions
(GGE) have still increased considerably across years. Although the
environmental consciousness-raising had already gained ground, according
to (``Luxembourg Sustainability Forum 2017 - Jean Jouzel, Les Enjeux Du
Réchauffement Climatique'' 2017) human being have to act now if he we
want to have a chance to reduce effects of climate change.

For the last several decades, companies have been more and more
considered as entities responsible for stewardship of the natural
environment (Majumdar and Marcus 2001; J. Przychodzen and Przychodzen
2015). Ecosystem degradation and resources depletion engender a threat
to firm's longevity (Dowell, Hart, and Yeung 2000), and as a reaction,
firms have to pro-actively adopt an environmental strategy (S. L. Hart
1995). In his speech at Lloyds of London 2015, Mark Carney, Governor of
the Bank of England and Chair of the Financial Stability Board (FSB),
identified climate change as one of the most material threats to
financial stability (Elliott 2015). To this end, companies facing higher
risks associated to climate change are ones subject to greater
incentives to develop green strategies (Hoffman 2005). However, both
economic benefits and strategic opportunities deriving from sustainable
development are usually underestimated by managers and still too many
companies do not feel concerned about global warming (Berchicci and King
2007; S. L. Hart 1995). Moreover, according to Scarpellini, Valero-Gil,
and Portillo-Tarragona (2016), green projects are not common in
companies of many countries because of significant barriers and a
negligible culture of excluding sustainable development from an
organization's strategy. If we consider that people's actions reflect a
variable mix of altruistic motivation, material self-interest, and
social or self-image concerns (Bénabou and Tirole 2006), demonstrating
that green development is a significant interest for firms could be a
serious step forward in the fight against global warming.

\textbf{To be continued\ldots{}}

\FloatBarrier
\newpage

\section{Literature Review}\label{literature-review}

According to\ldots{}

\FloatBarrier
\newpage

\section{Hypotheses}\label{hypotheses}

Here are my hypotheses

\FloatBarrier
\newpage

\section{Methodology}\label{methodology}

Here is my methodology\ldots{}

\FloatBarrier
\newpage

\section{Data Description}\label{data-description}

This is my data\ldots{}

\FloatBarrier
\newpage

\section{Results}\label{results}

Some incredible results\ldots{}

\FloatBarrier
\newpage

\section{Discussion}\label{discussion}

Let's speak\ldots{}

\FloatBarrier
\newpage

\section*{Conclusion}\label{conclusion}
\addcontentsline{toc}{section}{Conclusion}

This is my conclusion\ldots{}

\newpage

\section*{References}\label{references}
\addcontentsline{toc}{section}{References}

\hypertarget{refs}{}
\hypertarget{ref-Berchicci11PostcardsEdge2007}{}
Berchicci, Luca, and Andrew King. 2007. ``11 Postcards from the Edge.''
\emph{The Academy of Management Annals} 1 (1): 513--47.
doi:\href{https://doi.org/10.1080/078559816}{10.1080/078559816}.

\hypertarget{ref-BenabouIncentivesProsocialBehavior2006}{}
Bénabou, Roland, and Jean Tirole. 2006. ``Incentives and Prosocial
Behavior.'' \emph{The American Economic Review} 96 (5): 1652--78.
doi:\href{https://doi.org/10.1257/000282806779396283}{10.1257/000282806779396283}.

\hypertarget{ref-Dowell2000}{}
Dowell, Glen, Stuart Hart, and Bernard Yeung. 2000. ``Do Corporate
Global Environmental Standards Create or Destroy Market Value?''
\emph{Management Science} 46 (8): 1059--74.

\hypertarget{ref-Elliott2015}{}
Elliott, Larry. 2015. ``Carney Warns of Risks from Climate Change
'Tragedy of the Horizon'.'' \emph{The Guardian}. September 29.
\url{http://www.theguardian.com/environment/2015/sep/29/carney-warns-of-risks-from-climate-change-tragedy-of-the-horizon}.

\hypertarget{ref-HartNaturalResourceBasedViewFirm1995}{}
Hart, Stuart L. 1995. ``A Natural-Resource-Based View of the Firm.''
\emph{Academy of Management Review} 20 (4): 986--1014.
doi:\href{https://doi.org/10.5465/AMR.1995.9512280033}{10.5465/AMR.1995.9512280033}.

\hypertarget{ref-HoffmanClimateChangeStrategy2005}{}
Hoffman, Andrew J. 2005. ``Climate Change Strategy: The Business Logic
Behind Voluntary Greenhouse Gas Reductions.'' \emph{California
Management Review} 47 (3): 21--46.
doi:\href{https://doi.org/10.2307/41166305}{10.2307/41166305}.

\hypertarget{ref-HoughtonClimateChange19951996}{}
Houghton, John T., and Intergovernmental Panel on Climate Change. 1996.
\emph{Climate Change 1995: The Science of Climate Change: Contribution
of Working Group I to the Second Assessment Report of the
Intergovernmental Panel on Climate Change}. Cambridge University Press.

\hypertarget{ref-JeanJouzel2017}{}
``Luxembourg Sustainability Forum 2017 - Jean Jouzel, Les Enjeux Du
Réchauffement Climatique.'' 2017.

\hypertarget{ref-MajumdarRulesDiscretionProductivity2001}{}
Majumdar, Sumit K., and Alfred A. Marcus. 2001. ``Rules Versus
Discretion: The Productivity Consequences of Flexible Regulation.''
\emph{Academy of Management Journal} 44 (1): 170--79.
doi:\href{https://doi.org/10.2307/3069344}{10.2307/3069344}.

\hypertarget{ref-PrzychodzenRelationshipsecoinnovationfinancial2015}{}
Przychodzen, Justyna, and Wojciech Przychodzen. 2015. ``Relationships
Between Eco-Innovation and Financial Performance Evidence from Publicly
Traded Companies in Poland and Hungary.'' \emph{Journal of Cleaner
Production} 90 (March): 253--63.
doi:\href{https://doi.org/10.1016/j.jclepro.2014.11.034}{10.1016/j.jclepro.2014.11.034}.

\hypertarget{ref-Scarpellinieconomicfinanceinterface2016}{}
Scarpellini, Sabina, Jesús Valero-Gil, and Pilar Portillo-Tarragona.
2016. ``The `Economicfinance Interface' for Eco-Innovation Projects.''
\emph{International Journal of Project Management} 34 (6): 1012--25.
doi:\href{https://doi.org/10.1016/j.ijproman.2016.04.005}{10.1016/j.ijproman.2016.04.005}.


\end{document}
