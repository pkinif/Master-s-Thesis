\documentclass[]{article}
\usepackage{lmodern}
\usepackage{amssymb,amsmath}
\usepackage{ifxetex,ifluatex}
\usepackage{fixltx2e} % provides \textsubscript
\ifnum 0\ifxetex 1\fi\ifluatex 1\fi=0 % if pdftex
  \usepackage[T1]{fontenc}
  \usepackage[utf8]{inputenc}
\else % if luatex or xelatex
  \ifxetex
    \usepackage{mathspec}
  \else
    \usepackage{fontspec}
  \fi
  \defaultfontfeatures{Ligatures=TeX,Scale=MatchLowercase}
\fi
% use upquote if available, for straight quotes in verbatim environments
\IfFileExists{upquote.sty}{\usepackage{upquote}}{}
% use microtype if available
\IfFileExists{microtype.sty}{%
\usepackage{microtype}
\UseMicrotypeSet[protrusion]{basicmath} % disable protrusion for tt fonts
}{}
\usepackage[margin=1in]{geometry}
\usepackage{hyperref}
\hypersetup{unicode=true,
            pdftitle={ReflexionModel},
            pdfauthor={Kinif Pierrick},
            pdfborder={0 0 0},
            breaklinks=true}
\urlstyle{same}  % don't use monospace font for urls
\usepackage{graphicx,grffile}
\makeatletter
\def\maxwidth{\ifdim\Gin@nat@width>\linewidth\linewidth\else\Gin@nat@width\fi}
\def\maxheight{\ifdim\Gin@nat@height>\textheight\textheight\else\Gin@nat@height\fi}
\makeatother
% Scale images if necessary, so that they will not overflow the page
% margins by default, and it is still possible to overwrite the defaults
% using explicit options in \includegraphics[width, height, ...]{}
\setkeys{Gin}{width=\maxwidth,height=\maxheight,keepaspectratio}
\IfFileExists{parskip.sty}{%
\usepackage{parskip}
}{% else
\setlength{\parindent}{0pt}
\setlength{\parskip}{6pt plus 2pt minus 1pt}
}
\setlength{\emergencystretch}{3em}  % prevent overfull lines
\providecommand{\tightlist}{%
  \setlength{\itemsep}{0pt}\setlength{\parskip}{0pt}}
\setcounter{secnumdepth}{0}
% Redefines (sub)paragraphs to behave more like sections
\ifx\paragraph\undefined\else
\let\oldparagraph\paragraph
\renewcommand{\paragraph}[1]{\oldparagraph{#1}\mbox{}}
\fi
\ifx\subparagraph\undefined\else
\let\oldsubparagraph\subparagraph
\renewcommand{\subparagraph}[1]{\oldsubparagraph{#1}\mbox{}}
\fi

%%% Use protect on footnotes to avoid problems with footnotes in titles
\let\rmarkdownfootnote\footnote%
\def\footnote{\protect\rmarkdownfootnote}

%%% Change title format to be more compact
\usepackage{titling}

% Create subtitle command for use in maketitle
\newcommand{\subtitle}[1]{
  \posttitle{
    \begin{center}\large#1\end{center}
    }
}

\setlength{\droptitle}{-2em}
  \title{ReflexionModel}
  \pretitle{\vspace{\droptitle}\centering\huge}
  \posttitle{\par}
  \author{Kinif Pierrick}
  \preauthor{\centering\large\emph}
  \postauthor{\par}
  \predate{\centering\large\emph}
  \postdate{\par}
  \date{27 mars 2018}


\begin{document}
\maketitle

\begin{quote}
There has been a lack of consensus and norms in empirical studies
regarding the selection of the appropriate environmental performance,
financial performance, and control variables necessary to examine these
relationships (Wagner 2007).
\end{quote}

\section{What motivation to study the link between CFP and
CEP?}\label{what-motivation-to-study-the-link-between-cfp-and-cep}

\begin{quote}
From {[}@Dixon-Fowler2013a{]} : The primary arguments in this line of
research are that positive environmental performance represents a focus
on innovation and operational efficiency (e.g., Porter and van der Linde
1995), reflects strong organizational and management capabilities (e.g.,
Aragon-Correa 1998), enhances firm legitimacy (e.g., Hart 1995), and
allows a firm to meet the needs of diverse stakeholders (e.g., Freeman
and Evan 1990). First, environmental performance is viewed as a proxy
for operational efficiency (e.g., Porter and van der Linde 1995; Starik
and Marcus 2000). The ecoefficiency argument is based on the notion that
pollution is a waste of resources and represents unnecessary costs for
the firm (Porter and van der Linde 1995). Improved efficiency via
environmental performance lowers costs and increases innovation leading
to competitive advantage (Aragon-Correa 1998;Christmann 2000;Judge and
Douglas 1998; Klassen and Whybark 1999; Russo and Fouts 1997;
Shrivastava 1995). Second, strong environmental performance might be
viewed as a measure of organizational and managerial capabilities
including a long-term versus short-term perspective, a focus on
continuous innovation and reduced organizational risk (Aragon-Correa
1998; Hart 1995; Sharma 2000; Russo and Fouts 1997; Sharma and
Vredenburg 1998; Shrivastava1995). Third, firms with strong
environmental performance might reap reputational benefits, which result
in social legitimacy (Hart 1995), the ability to attract and retain
quality employees (Turban and Greening 1997), and increased sales (Russo
and Fouts 1997). Finally, instrumental stakeholder theory posits that to
be successful, firms must meet the needs of diverse stakeholder groups,
including environmental, employee, and societal groups (Freeman and Evan
1990; Marcus and Geffen 1998; Sharma and Vredenburg 1998).
\end{quote}

\section{Deep analysis of {[}@Chencrosscountrycomparisongreen2018{]} and
{[}@LiUnderstandingImpactGreen2017{]}}\label{deep-analysis-of-chencrosscountrycomparisongreen2018-and-liunderstandingimpactgreen2017}

\begin{itemize}
\tightlist
\item
  Hypothesis 1: The higher the level of green initiatives (Pay Link,
  Sustainability Themed Committee and Audit), the higher the level of
  green performance (Energy Productivity, Carbon Productivity, Water
  Productivity, Waste Production and Green Revenue).
\end{itemize}

The first hypothesis have been tested with T-tests on the impact of each
green initiative on green performance

\begin{itemize}
\item
  Hypothesis 2: The higher the level of green performance (Energy
  Productivity, Carbon Productivity, Water Productivity, Waste
  Production and Green Revenue), the higher the level of financial
  performance (Debit Ratio, Profit Margin, Return on Assets, Market to
  Book Ratio and Assets Turnover).
\item
  Hypothesis 3: The higher the level of green initiatives (Pay Link,
  Sustainability Themed Committee and Audit), the higher the level of
  financial performance (Leverage, Profit Margin, Return on Assets,
  Market to Book Ratio and Assets Turnover).
\end{itemize}

Hypothesis 2 and 3 were tested by regression analysis. Five regression
analysis were conducted to see how green initiatives and green
performance impact financial performance using Energy Productivity,
Carbon Productivity, Water Productivity, Waste Productivity, Green
Revenue, Pay Link, Sustainability Themed Committee and Audit as
independent variables, and each financial indicator (Leverage, Profit
Margin, Return on Total Assets, Market to Book Ratio and Assets
Turnover) as dependent variables

Les auteurs ont donc réalisé 5 régressions pour tester les hypothèses
deux et trois. Ils ont pris chaque indicateurs financiers comme VD
qu'ils ont régressé sur les variables de green initiatives et
performances.

\textbf{Dans le contexte d'une panel data analysis, je ne pense pas que
ce soit pertinent de faire ainsi. Aussi, quel est l'intérêt d'avoir
plusieurs VD? Pourquoi ne pas prendre qu'une seule VD comme indicateur
des performances financiers? Et genre ROA, ROE, leverage ect\ldots{}
variable de controle?
{[}@PrzychodzenRelationshipsecoinnovationfinancial2015{]}}

\section{Corporate Financial Performance as Dependent
variable}\label{corporate-financial-performance-as-dependent-variable}

\subsection{{[}@Dixon-Fowler2013a{]}}\label{dixon-fowler2013a}

{[}@Dixon-Fowler2013a{]} carried out a meta analytics review to adress
the question ``When does it pay to be green?''. {[}@Albertini2013{]} dit
the same and even more in answering this question ``How does it pay to
be green?''.

\emph{{[}Dixon-Fowler2013a{]} provide a meta-analytic review of CEP--CFP
literature in which they identify potential moderators to the CEP--CFP
link including environmental performance type (reactive vs.~proactive),
firm characteristics (e.g., large vs.small firm, public vs.~private
firm, US-based firms vs.international firms, and worst offenders vs.~a
broader.}

\textbf{The following citation come to reinforce my conviction to test
the hypotheseaccording to the industry sector.}

\begin{quote}
From Dixon-Fowler2013a{]}: Regulatory differences for firms in certain
industries may also influence the relationship between CEP and CFP.
Specifically, it is possible that the ``worst offenders,'' firms in
industries with negative reputations regarding environmental
performance, may experience greater media attention and more pressure
from NGOs, consumers, and governmental authorities, resulting in the
potential for greater gains in organizational legitimacy through better
environmental performance (Bansel 2005; Berrone and Gomez-Mejia 2009;
Hoffman 2001). Moreover, executives in such high polluting industries,
for example, may have less influence over the environmental performance
given the nature of the business (Berrone and Gomez-Mejia 2009). A
number of existing CEP--CFP studies have focused on the ``worst
offenders,'' particularly high polluting industries (i.e., oil and gas,
heavy manufacturing, EPA lists, etc.) (e.g., Bragdon and Marlin 1972;
Christmann 2000; Clarkson et al.2008; Freedman and Jaggi 1986).
\end{quote}

\subsection{{[}@Albertini2013{]}}\label{albertini2013}

Below is copy-paste only :

\begin{itemize}
\tightlist
\item
  According to her : Financial Performance Variables. Financial
  performance is a meta-construct emphasizing the profitability of the
  firm. The studies in this meta-analysis have mainly adopted three
  broad subdivisions of CFP: market-based (investor returns),
  accounting-based (accounting returns), and organizational measures.
\end{itemize}

\begin{enumerate}
\def\labelenumi{\arabic{enumi}.}
\item
  \textbf{Accounting-based indicators} often use earnings per share
  (EPS), return on equity (ROE), return on assets (ROA), return on sales
  (ROS), and return on investment (ROI) to measure the financial
  performance of the firm. ROA and ROE are generally accepted standard
  measures of financial performance found in research on strategy. In
  addition to ROA or ROE, Tobin's q reflects the inherent value of the
  firm and the expected future gains in accordance with `` {[}it{]} pays
  to be green'' studies (Dowell et al., 2000). Accounting-based
  indicators are subject to managers' discretionary allocations of funds
  to different project choices. They reflect internal decisionmaking
  capabilities and managerial performance rather than external market
  responses to organizational (nonmarket) actions (Cochran \& Wood,
  1984).
\item
  Other studies used \textbf{market-based indicators} such as a
  price-earning ratio, price per share, or share price appreciation
  (Orlitzky, 2005) to underline the improvement of the firm's economic
  performance. These measures focus only on the economic performance of
  a firm without taking into account the specific consequences of
  pro-environmental strategies on financial performance. Market-based
  indicators are said to be subject to forces beyond management's
  control. These indicators reflect the notion that shareholders are a
  primary stakeholder group whose satisfaction determines the company's
  fate (Grossman \& Hoskisson, 1998).
\item
  Furthermore, CEM involves \textbf{organizational processes} measured
  by other indicators than accounting-based or market-based indexes.
  Cost advantage generated by pollution control equipment, or
  differentiation advantage due to green product sales or due to a
  firm's good environmental reputation are used to measure financial
  performance (Christmann, 2000). Hart (Hart, 1995; Hart \& Dowell,
  2011) and Porter and van der Linde (1995a) have presented the argument
  of innovation offsets and specific capabilities developed by proactive
  companies. Green product innovation and environmental process
  innovation are seen to be good proxies for the evaluation of
  competitive advantage in natural resource--based view research (Judge
  \& Douglas, 1998).
\end{enumerate}

\begin{itemize}
\item
  conclude that the relationship between CEM and CFP is more positive
  when CFP is measured by accounting-based indicators than when it is
  measured by other financial performance variables.. The results
  confirm theoretical inconsistencies (stakeholder mismatching), as the
  relationship between CEM and CFP is not significant when CFP is
  measured by market-based indicators. As Wood and Jones (1995) argue,
  there is no theory that explains whether stakeholders would or would
  not prefer a company that invests money in green issues to be ranked
  higher in pollution control indexes.
\item
  Firms that have addressed the environmental issue by implementing an
  EMS need to develop particular organizational capabilities. This type
  of environmental management transforms the organization, modifying
  manufacturing processes and integrating environmental management into
  day-to-day operations. Environmental performance improvement becomes
  an objective of a firm's strategy, just as financial profitability is.
  The goal of an environmentally proactive strategy is to significantly
  reduce pollution through well-defined environmental objectives rather
  than to merely control emissions through end-of-pipe ments. This kind
  of strategy is people-intensive and depends on tacit skill development
  through employee involvement (Hart, 1995; Hart \& Dowell, 2011).
\end{itemize}

\begin{quote}
In this context it seems that ``How does it pay to be green?'' is a more
accurate question than ``Does it pay to be green?'' when addressing the
consequences of CEM on CFP (Kallio \& Nordberg, 2006). Our results
confirm that the organization and environment-related research is today
legitimate without the need to find win--win solutions.
\end{quote}

\textbf{ATTENTION}

\emph{Albertini 2013 have inserted an appendix at the end of her paper.
This one giver a strong overview of studies included in the
meta-analyses. Therefore I can know for each papers which indicators
they have used for both CEP and CFP !!!!}

\textbf{Based on this table} : I can observe that some papers are
interesting for my research. A lot used ROA and tobin's Q.

\begin{itemize}
\item
  \textbf{{[}@Berrone2009{]}} used ROE and Tobin's Q as CFP indicators
  and both chemical emission and pollution level (waste generation
  level) as indicators of CEP. They made a longitudinal study
  (i.e.~Fixed-Effects and Random effect Panel Data Analyses depending on
  the hypotheses) with a sample of 469 US firms giving a total of 2088
  observations. They also used firm size as a control variable.
\item
  \textbf{{[}@Busch2011{]}} --\textgreater{} do not have acces
\item
  \textbf{{[}@Sulkowski2011{]}} --\textgreater{} do not have access, I
  have asked for it, let's wait a bit.
\item
  \textbf{{[}@WahbaHayam2007{]}} --\textgreater{} do not have access, I
  have asked for it, let's wait a bit.
\item
  \textbf{{[}@Nakao2007{]}} --\textgreater{} do not have access, I have
  asked for it, let's wait a bit.
\item
  \textbf{{[}@Wu2010{]}} --\textgreater{} do not have access, I have
  asked for it, let's wait a bit.
\item
  \textbf{{[}@Makni2009{]}} --\textgreater{} This study assesses the
  causal relationship between corporate social performance (CSP) and
  financial performance (FP). We perform our empirical analyses on a
  sample of 179 publicly held Canadian firms and use the measures of CSP
  provided by Canadian Social Investment Databasefor the years 2004 and
  2005. Using the ``Granger causality'' approach, we find no significant
  relationship between a composite measure of a firm's CSP and FP,
  except for market returns. However, using individual measures of CSP,
  we find a robust significant negative impact of the environmental
  dimension of CSP and three measures of FP, namely return on assets,
  return on equity, and market returns. They used 3 control variable :
  size, industry sector and risk.
\item
  \textbf{{[}@King2001{]}} : Explore the link between CFP and CEP in
  analyzing 652 U.S. manufacturing firms over the time period 1987 to
  1996 (i.e.~longitudinal study - OLS, fixed effect model and random).
  Financial indicators are ROA and Tobin's Q. Environmental indicators
  are TRI and pollution reduction.
\end{itemize}

\begin{quote}
\textbf{Dependent Variable} \ldots{} In accordance with more recent
``pays to be green'' studies, we use a simplified measure of Tobin's q
(Dowell, Hart, \& Yeung 2000).We calculate Tobin's q by dividing the sum
of firm equity value, book value of long-term debt, and net current
liabilities by the book value of total assets.All financial data were
obtained from the Compustat database.
\end{quote}

\begin{quote}
\textbf{Variable control:} These measures include 1) the company's size
(Firm Size) calculated as the log of the company's assets, 2) the
capital intensity of a firm (Capital Intensity) calculated by dividing
capital expenditures by sales, 3) the annual growth of the firm (Growth)
calculated as the percentage change in sales, 4) the degree to which the
firm is leveraged (Leverage) divided as the ratio of its debt to assets,
and 5) the R\&D intensity (R\&D Intensity) calculated by dividing
research \& development expenses by total assets.
\end{quote}

\begin{quote}
\textbf{Conclusion}: In this paper, we further explore whether it ``pays
to be green''. We use longitudinal data and statistical methods that
reduce the potential for unobserved differences among firms to create a
misleading association between environmental and financial performance.
We also test to see whether pollution reduction causes financial gain.
Table 5 presents a summary of these results. We find evidence of an
association between pollution reduction and financial gain, but we
cannot prove the direction of causality. We also show that firms in
cleaner industries have higher Tobin's q, but we are unable to rule out
possible confounding effects from fixed firm attributes. Moreover, we
cannot show that firms that move to cleaner industries improve their
financial performance.
\end{quote}

\textbf{I think I should really test if industry sector impact the
relationship between CFP and CEP}

\begin{itemize}
\tightlist
\item
  \textbf{{[}@Dowell2000{]}} : Longitudinal study with S\&P 500 listed
  companies from 1994 to 1997. Study the link between CFP(i.e; tobin's
  q) and CEP (i.e.~IRRC Corporate environmental profile data) with some
  control variables (i.e.~capital structure, R\&D and avertising
  expenditues, multinationality, leverage, firm size).
\end{itemize}

\subsection{Ludecasdedebatnexus2014}\label{ludecasdedebatnexus2014}

\begin{itemize}
\tightlist
\item
  {[}@Ludecadedebatenexus2014{]} have presented a critical review of
  relevant empirical research articles on the nexus between corporate
  social performance (i.e.~CSP) and corportate financial performance
  (i.e.~CFP) published during the ten-year period from 2002 to 2011.
  They decomposed both variables into several group classification and
  summarized the frequencies of appearance of control variables.CFP had
  been subdivided into the three-group classification of
  {[}@Orlitzky2003{]}, namely :
\end{itemize}

\begin{enumerate}
\def\labelenumi{\arabic{enumi}.}
\item
  \textbf{the accounting-based measure} which consists of profitability
  measures, asset utilization (e.g.~Return on asset, Asset Turnover) and
  growth measures {[}@Orlitzky2003{]}.
\item
  \textbf{the market-based measure} reflect the notion that shareholders
  are a primary stackeholder group {[}@Cochran1984{]} and contains
  several measures such as price per share, share price appreciation,
  stock performance, market return, market value to book value and
  others {[}@VanBeurden2008{]}.
\item
  and \textbf{the perceptual measure} of CFP which ask survey
  respondents to provide subjective estimates of firms' financial
  performance.
\end{enumerate}

\begin{quote}
Amongst the three generic CFP measures, the accounting-based measures
are objective and audited, market-based measures are partly objective,
and perceptual are largely subjective based on the survey respondents
perceptions {[}@Ludecadedebatenexus2014{]}.
\end{quote}

\begin{quote}
\ldots{} weaknesses in one type of performance measure can be alleviated
to some extent by the use of another {[}@McWilliams2006{]}
\end{quote}

Thus {[}@Ludecadedebatenexus2014{]} have shown that 21 empirical studies
of their sample, namely 81 papers, have used two types of CFP measures
in their own paper.

\subsection{Delmas 2015}\label{delmas-2015}

\begin{itemize}
\tightlist
\item
  {[}@Delmas2015{]}: Existing studies commonly use accounting- or
  market-based measures of financial performance interchangeably
  (Margolis et al., 2007; Peloza, 2009). However, both methods are not
  perfect substitutes. Accounting measures are often used to evaluate
  initiatives that affect the firm in the short term, such as those that
  reduce operating costs (Peloza, 2009). In contrast, market-based
  measures capture investors' long-term perceptions of the future
  profitability of a firm's current or recent management practices
  (Dowell et al., 2000; King \& Lenox, 2002; Konar \& Cohen, 2001).
\end{itemize}

\begin{quote}
We use ROA and Tobin's q to approximate short- and long-term
perspectives of financial performance, respectively. We calculate these
variables based on financial information provided by Compustat. ROA is a
standard accounting measure of financial performance, which is
calculated by dividing earnings before interest by total assets (King \&
Lenox, 2002). Tobin's q is defined as the ratio of a firm's market value
to the replacement cost of its assets, which this study approximates
using the method developed in Chung and Pruitt (1994), which we describe
in Appendix A. Tobin's q incorporates the market value of firms and is
thus able to reflect intangible attributes, which are not captured by an
accounting-based measure like ROA. ROA and Tobin's q provide
complementary information regarding a firm's financial performance,
which allows us to differentially evaluate the effect of environmental
performance. Whereas the former demonstrates how efficiently a firm
generates profit per unit of production,the latter reflects intangible
measures of performance, like investor confidence and reputation (Dowell
et al., 2000; King \& Lenox, 2002; Konar \& Cohen, 2001). In this sense,
Tobin's q can incorporate how robust the market interprets a firm to be
in the face of future climate legislation, whereas ROA only acknowledges
a firm's GHG emissions indirectly via the efficiency of its use in
producing earnings (Busch \& Hoffmann, 2011). Both measures have been
used in empirical research into the effect of environmental performance
on financial performance (Dowell et al., 2000; Elsayed \& Paton, 2005;
King \& Lenox, 2002). However, to the best of our knowledge, only King
and Lenox (2002) and Nakao, Amano, Matsumura, Genba, and Nakano (2007)
used both measures in the same study. Notably, both studies uncovered
substantively equivalent effects of environmental performance on Tobin's
q and on ROA. Compared to ROA, calculating Tobin's q requires a
relatively high number of financial variables and is more susceptible to
missing values. This creates a discrepancy in the number of observations
for each dependent variable in this study, resulting in asymmetric
sample spaces (see Table 4). To check whether this introduces sample
bias, an identical analysis was conducted on the set of observations
common to both dependent variables. The results were robust to both
sample spaces (results available on request from the authors).
{[}@Delmas2015{]}
\end{quote}

\begin{quote}
Concerning dependent variable, extent research on CFP widely uses either
market-based indicators that capture firm's current and future gains
(Tobin's q, market capitalization, stock market performance), and/or
accounting-based indicators (i.e., return on equity (ROE), ROA, return
on sales (ROS)) that capture a firm's current ability to create value by
using its assets (Sanchez-Ballesta and García-Meca, 2007). Tobin's q
and ROE have been used to account for both market- and accounting-based
measures of CFP. Tobin's q is estimated as the ratio between (book value
of total assets e book value of shareholder's equity þ market value of
shareholder's equity) and (book value of total assets). ROE is defined
as (net income e preferred dividend requirements) divided by the average
of last year's and current year's common equity. To reduce the weight of
extreme outliers, Tobin's q and ROE have been winsorized at the 5th and
95th percentiles.{[}@ManriqueAnalyzingEffectCorporate2017{]}
\end{quote}

\section{Corporate Social
Performance}\label{corporate-social-performance}

\subsection{Corporate Environmental
Management}\label{corporate-environmental-management}

\begin{itemize}
\tightlist
\item
  {[}@Albertini2013{]} considers corporate environmental management
  (CEM) as a concept that embraces environmental management,
  environmental disclosure, and environmental performance.
\end{itemize}

\begin{quote}
\ldots{} as research in the area of environmental management has
increased, various proxies have been utilized to measure CEM (Walls,
Phan, \& Berrone, 2011; Yu et al., 2009). Most studies have used
negative externalities (Earnhart \& Lizal, 2007; Hart \& Ahuja, 1996);
some have used data reported from the Toxic Release Inventory (Clarkson
\& Li, 2004; Dooley \& Lerner, 1994; Hamilton, 1995); some have used
voluntary participation in environmental programs (Dowell, Hart, \&
Yeung, 2000; Khanna \& Damon, 1999); yet others have used rewards or
similar recognition (Klassen \& McLaughlin, 1996) or environmental
disclosure (Blacconiere \& Northcut, 1997; Cohen, Fenn, \& Konar, 1997).
These different proxies focus on specific goals of CEM practices such as
maintaining legitimacy, conforming to environmental regulation, or
improving environmental performance. Thus, a close examination of
research findings is critical for furthering knowledge in this
area.{[}@Albertini2013{]}
\end{quote}

\subsection{Definition of CSP and link with green performance and
sustainable
development?}\label{definition-of-csp-and-link-with-green-performance-and-sustainable-development}

\subsection{CSP classification}\label{csp-classification}

\section{Control Variables}\label{control-variables}

\begin{itemize}
\item
  As seen from Table 4, size, industry, capital structure, financial
  return (ROA, ROE, ROS and EPS) and risk are the top five most
  frequently used control variables in explaining the CSP-CFP
  relationship. This largely confirms Andersen and Dejoy's (2011)
  summary that size, industry, risk, R\&D and advertising expenses are
  the most commonly used control variables.
\item
  {[}@PrzychodzenRelationshipsecoinnovationfinancial2015{]} use Fiancial
  leverage (i.e.~debt ratio as control variale) to explore four types of
  eco-innovation (product, process, market and sources of supply) and
  their impact on accounting-based measurers of financial performance
  using the data on Polish and Hungarian publicly traded companies from
  the years 2006e2013
\item
  {[}@Delmas2015{]} use includes several financial variables to control
  for sources of firm-level heterogeneity, in line with previous studies
  of financial and environmental performance (Dowell et al., 2000;
  Elsayed \& Paton, 2005; King \& Lenox, 2002). Firm total assets
  account for variation in firm size, while leverage is approximated by
  the ratio of total debt to total assets. Growth is defined as the
  annual change in sales divided by total sales and controls for
  variations in production (King \& Lenox, 2002). Capital expenditures
  divided by total sales controls for capital intensity (Elsayed \&
  Paton, 2005; King \& Lenox, 2002). Due to a prohibitively large number
  of missing values for research and development expenditures in the
  Compustat database, this variable was not included in our analysis
  (see McWilliams \& Siegel, 2000). \textbf{To correct for skewed
  distributions, the financial control variables are transformed using
  the natural logarithm.}
\end{itemize}

\textbf{Pourquoi parfois ROA = Control Variable et parfois = VD? Trouver
justification de pourquoi j'utilise ROA en VD plutot que VC? Ou
inverse?}

\begin{itemize}
\item
  As control variables, in all the models financial leverage, sales
  growth, firm size, country, industry and year dummies have been
  included. Given that firms with high indebtedness experience
  significant financial constraints and, ultimately, deliver inferior
  CFP (Gleason et al., 2000), the leverage ratio is included in the
  present study as a proxy of financial distress (estimated as the ratio
  of total debt to total assets). There is also a vast amount of
  research that suggests that sales growth has a positive influence on
  firm profitability (Delmar et al., 2013). The log-difference of net
  sales for firm i between time t and t-1 is adopted as a proxy of the
  sales growth (García-Manjon and Romero-Merino, 2012). The natural
  logarithm of total assets has been included in all the regressions to
  control for the effect of firm size (Becker-Blease et al., 2010).
  Country, industry and year dummies have been included in order to
  capture the heterogeneity across different countries, industrial
  sectors and time periods. {[}@ManriqueAnalyzingEffectCorporate2017{]}
\item
  {[}@Albertini2013{]} have tested if both sector industry and duration
  of the empirical study moderates the relationship between CEM and CFP.
  They concluded that the duration of the empirical study strongly
  moderates the relationship between CEM and CFP so that it is
  significantly positive and stronger for non-longitudinal studies. They
  also found that the same relationship is not significantly moderated
  by the studies'activity sector.
\end{itemize}

\section{Relation between CEM/CSP and
CFP}\label{relation-between-cemcsp-and-cfp}

\begin{itemize}
\tightlist
\item
  From {[}@Albertini2013{]} : A number of studies have proposed
  conceptual frameworks or explanations for the existence of a causal
  relationship between CEM and CFP. Numerous studies have suggested that
  the relationship is positive (Al Tuwaijri, Christensen, \& Hughes,
  2004; Hart \& Ahuja, 1996; Judge \& Douglas, 1998; Montabon, Sroufe,
  \& Narisimhan, 2007; Russo \& Fouts, 1997; Sroufe, 2003; Stanwick \&
  Stanwick, 1999) following Porter's ``win-win'' argument and natural
  resource--based view theory (Hart, 1995; Hart \& Dowell, 2011).
  However, other research has concluded that CFP is negatively
  associated with CEM over a short period of time (Blacconiere \&
  Patten, 1994; Jaggi \& Freedman, 1992); and with proactive
  environmental strategies over a longer period of time (Cordeiro \&
  Sarkis, 1997; McPeak, Devirian, \& Seaman, 2010; Yu, Ting, \& Wu,
  2009). Yet other studies have established that the relationship
  between CEM and CFP cannot be proved because of the difficulties of
  measuring the environmental management consequences on profitability
  (Collison, Lorraine, \& Power, 2004; King \& Lenox, 2001; Murray,
  Sinclair, Power, \& Gray, 2006).
\end{itemize}

\begin{quote}
Even though the results of previous empirical research remain
contradictory, the relationship between CEM and CFP seems to be
positive. Therefore the research question worth studying is ``When does
it pay to be green?'' rather than ``Does it pay to be green?''
{[}@Albertini2013{]}
\end{quote}

\section{References}\label{references}


\end{document}
